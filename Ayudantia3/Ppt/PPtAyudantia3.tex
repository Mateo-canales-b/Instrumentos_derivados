\documentclass{beamer}

\usetheme{Darmstadt}
\usepackage{comment}
\usepackage[utf8]{inputenc}    % Para escribir acentos y caracteres especiales
\usepackage{graphicx}          % Para insertar imágenes
\usepackage{booktabs}          % Para tablas más bonitas
\usepackage{enumitem}
\usepackage{amsmath}           % Para escribir ecuaciones
\usepackage{tikz}              % Para gráficos vectoriales
\usetikzlibrary{calc}          % Para coordenadas calculadas con TikZ
\usepackage{pgfplots}          % Para gráficos matemáticos
\pgfplotsset{compat=1.18}      % Evita errores por compatibilidad
\usepackage{hyperref}
\usepackage{pgf}
\usepackage{ragged2e}

\newif\ifpresentacion
\presentaciontrue  % Cambiar a \presentacionfalse para PDF sin animaciones
%\presentacionfalse % Cambiar a \presentaciontrue para PDF con  animaciones

\usepackage{ifthen}
\newcommand{\pausa}{\ifpresentacion\pause\fi}
% #region Título
\title{Ayudantía 3 \\ Valoración de Futuros y
Forwards \\ \large\textit{Instrumentos Derivados}}
\author{
  \texorpdfstring{
    \textbf{Profesor:} Francisco Rantul \\[0.3em]
    \textbf{Ayudante:} Mateo Canales
  }{Profesor: Francisco Rantul, Ayudante: Mateo Canales}
}
\subject{Instrumentos Derivados}
\institute{Universidad Diego Portales}
\date{16 De Abril, 2025}
% #endregion 

\begin{document}

% Portada
\begin{frame}
    \titlepage
    \vfill
    \centering
    \includegraphics[width=2.3118cm]{../imagenes/logo.png}
  \end{frame}

% #region Formateo de números 
\newcommand{\cajaverde}[1]{\fcolorbox{blue}{green!20}{#1}}
\newcommand{\entero}[1]{\pgfmathprintnumber[fixed, precision=0]{#1}}
\newcommand{\decimal}[1]{\pgfmathprintnumber[fixed, precision=2]{#1}}
\newcommand{\decimalx}[1]{\pgfmathprintnumber[fixed, precision=3]{#1}}
\newcommand{\decimalxx}[1]{\pgfmathprintnumber[fixed, precision=4]{#1}}
\newcommand{\porcentaje}[1]{%
  \pgfmathsetmacro{\temp}{#1*100}%
  \pgfmathprintnumber[fixed, precision=2]{\temp}\%%
  }
\newcommand{\dinero}[1]{%
  \$\,\pgfmathprintnumber[fixed, precision=3]{#1}
  }
\pgfkeys{/pgf/number format/.cd,1000 sep={\,}} % usa espacio fino como separador

\newcommand{\miles}[1]{\pgfmathprintnumber[fixed, precision=0]{#1}}
\newcommand{\forward}{\ensuremath{F_0 =  S_0 \cdot{e^{r \cdot T}}}}
\newcommand{\forwarddividendo}{\ensuremath{F_0 =  \left(S_0-I \right) \cdot{e^{r \cdot T}}}}
\newcommand{\forwarddividendoper}{\ensuremath{F_0 =  S_0 \cdot{e^{\left(r-q\right)\cdot T}}}}
\newcommand{\forwardalmacenamiento}{\ensuremath{F_0 =  \left(S_0+U \right) \cdot{e^{r \cdot T}}}}
\newcommand{\capcontinuacero}{ F = P \cdot e^{-rT}}
\newcommand{\forwardmonedas}{\ensuremath{F_0 =  S_0 \cdot{e^{\left(r-r_r\right) \cdot T}}}}

% #endregion

% #region Variables pregunta 4

\newcommand{\plata}{18.8}
\newcommand{\almacenamiento}{0.4}
\newcommand{\tlr}{0.04}
\newcommand{\pdolardos}{708}
\newcommand{\tclpdos}{0.0264}
\newcommand{\tusddos}{0.01}
\newcommand{\espflujouno}{\ensuremath{P = E \left(\sum_{i=1}^{n} Flujo_i \right) \cdot{e^{-r \cdot T}}}}
\newcommand{\tcuatroa}{\frac{9}{12}}
\pgfmathsetmacro{\tcuatroac}{9/12}
\newcommand{\tcuatroab}{\frac{6}{12}}
\pgfmathsetmacro{\tcuatroabc}{6/12}
\newcommand{\tcuatroaa}{\frac{3}{12}}
\pgfmathsetmacro{\tcuatroaac}{3/12}
\pgfmathsetmacro{\flujocuatroa}{\almacenamiento*exp(-\tlr*\tcuatroaac)}
\pgfmathsetmacro{\flujocuatrob}{\almacenamiento*exp(-\tlr*\tcuatroabc)}
\pgfmathsetmacro{\flujocuatroc}{\almacenamiento*exp(-\tlr*\tcuatroac)}
\pgfmathsetmacro{\flujocuatro}{\flujocuatroa+\flujocuatrob+\flujocuatroc}
\pgfmathsetmacro{\valoresecero}{\plata+\flujocuatro }
\pgfmathsetmacro{\expcuatro}{\tlr*\tcuatroac }
\pgfmathsetmacro{\finalcuatro}{\valoresecero * exp(\expcuatro)}

% #endregion

\section{Pregunta 4}

  \begin{frame}{Pregunta 4}
    \justify
    Suponga que el precio spot del commodity de plata es actualmente igual a \dinero{\plata} dólares  
    la onza. Los costos de almacenamiento son iguales a \dinero{\almacenamiento}por año la onza, pagaderos por
    trimestres vencidos. La estructura de tasas de interés es plana con una tasa cero libre 
    de riesgo del \porcentaje{\tlr} anual compuesto continuo.
    \begin{enumerate}[label=\textbf{\alph*)}]
    \item   Se le pide calcular cual debería ser el precio de futuros de plata, con entrega a 9 meses plazo.
    \item 	Explique que ocurre si el precio de futuros de plata con entrega a 9 meses,
    efectivamente observado en el mercado es de \$19,8 dólares la onza.
    \item 	Asuma que existe una baja demanda por el commodity de plata, por lo tanto, las empresas están 
    almacenando altos niveles de inventarios. Explique qué ocurre con el rendimiento de conveniencia.
    
  \end{enumerate}
  \end{frame}

  \subsection{Parte a)}

  \begin{frame}{Pregunta 4 parte a)}
    \justify

  Suponga que el precio spot del commodity de plata es actualmente igual a \dinero{\plata} dólares  
  la onza. Los costos de almacenamiento son iguales a \dinero{\almacenamiento} por año la onza, pagaderos por
  trimestres vencidos. La estructura de tasas de interés es plana con una tasa cero libre 
  de riesgo del \porcentaje{\tlr} anual compuesto continuo.\\
  \vspace{.5em}
  \textbf{a)} Se le pide calcular cual debería ser el precio de futuros de plata, con entrega a 9 meses plazo.

  \end{frame}

  \begin{frame}{Desarrollo a)}
    \justify

    Datos: \(S_0=\plata\), \(almacenamiento=\almacenamiento\), \(r=\tlr\), \(T=\tcuatroa\).\\
    \pausa
    Anteriormente usamos una fórmula para dividendos, y que si lo pensamos de cierta manera, 
    el almacenamiento es lo mismo que un dividendo pero el flujo es contrario.\\
    \pausa
    \vspace{.5em}
    \(\forwarddividendo\  \pausa \rightarrow\ \) \textcolor{blue}{\(\forwardalmacenamiento\)}\\
    \pausa
    Siendo U el valor presente del valor de almacenamiento. con la fórmula de valorización continua \footnote{ \(\capcontinuacero\)}.\\
    \pausa
    \(U=Alm_1 + Alm_2 + Alm_3\)\\
    \pausa
    \(U= \almacenamiento \cdot e^{-\tlr \cdot \tcuatroaa} + \almacenamiento \cdot e^{-\tlr \cdot \tcuatroab} + \almacenamiento \cdot e^{-\tlr \cdot \tcuatroa}\)\\
    \pausa
    \(U = \decimalx{\flujocuatroa} + \decimalx{\flujocuatrob} + \decimalx{\flujocuatroc}\)\\
    \pausa
    \cajaverde{
    \(U = \decimalx{\flujocuatro}\)
    } 
  \end{frame}

  \begin{frame}{Desarrollo a) (continuación)}
    \Large
    Datos: \(S_0=\plata\), \(r=\tlr\), \(T=\tcuatroa\),  \(U = \decimalx{\flujocuatro}\).\\
    \textcolor{blue}{\(\forwardalmacenamiento\)}\\
    \pausa
    Reemplazando\\
    \(F_0= \left(\plata + \decimalx{\flujocuatro}\right) \cdot e^{\tlr \cdot \tcuatroa } \)\\
    \pausa
    \(F_0= \left(\decimalx{\valoresecero}\right) \cdot e^{\decimalx{\expcuatro}} \)\\
    \pausa
    \cajaverde{
    \(F_0= \decimalx{\finalcuatro}\)
    }
  \end{frame}

  \subsection{Parte b)}

  \begin{frame}{Pregunta 4 parte b)}
    \justify
    \large
  Suponga que el precio spot del commodity de plata es actualmente igual a \dinero{\plata} dólares  
  la onza. Los costos de almacenamiento son iguales a \dinero{\almacenamiento} por año la onza, pagaderos por
  trimestres vencidos. La estructura de tasas de interés es plana con una tasa cero libre 
  de riesgo del \porcentaje{\tlr} anual compuesto continuo.\\
  \vspace{.5em}
  \textbf{b)} Explique que ocurre si el precio de futuros de plata con entrega a 9 meses,
  efectivamente observado en el mercado es de \$19,8 dólares la onza.

  \end{frame}

  \begin{frame}{Desarrollo b)}
  \justify
  \Large
  \textbf{Respuesta:}  
  Según la fórmula \(\forwardalmacenamiento\), el valor calculado debería ser igual al valor que está en el mercado,
  por ende, es correcto asumir que existirian posibilidades de arbitraje.
  \end{frame}

  \subsection{Parte c)}

  \begin{frame}{Pregunta 4 parte c)}
    \justify
    \large
    Suponga que el precio spot del commodity de plata es actualmente igual a \dinero{\plata} dólares  
    la onza. Los costos de almacenamiento son iguales a \dinero{\almacenamiento} por año la onza, pagaderos por
    trimestres vencidos. La estructura de tasas de interés es plana con una tasa cero libre 
    de riesgo del \porcentaje{\tlr} anual compuesto continuo.\\
    \vspace{.5em}
    \textbf{c)}	Asuma que existe una baja demanda por el commodity de plata, por lo tanto, 
    las empresas están almacenando altos niveles de inventarios. 
    Explique qué ocurre con el rendimiento de conveniencia. 
  \end{frame}

  \begin{frame}{Desarrollo c)}
    \justify
    \large
    Recordando, los rendimiento de conveniencia $(y)$ son los beneficios derivados de la tenencia del activo físico.
    Es decir, en algunos casos es recomendable mantener el activo como activo de consumo (materia prima) que como activo
    de inversión (futuro).\\
    \(F_0 \leq \left(S_0+U\right) \cdot e^{rT} \)\\
    \pausa
    Incluyendo el retorno de conveniencia:\\
    \(F_0 \cdot e^{yT}= \left(S_0+U\right) \cdot e^{rT} \)\\
    \pausa
    \(F_0 = \left(S_0+U\right) \cdot \frac{e^{rT}}{e^{yT}}  \)\\
    \pausa
    \(F_0 = \left(S_0+U\right) \cdot e^{\left(r-y\right)T} \)\\
  \end{frame}
    
  \begin{frame}{Desarrollo c) (continuación)}
    \justify
    \Large
    \(F_0 = \left(S_0+U\right) \cdot e^{\left(r-y\right)T} \)\\
  \vspace{1em} 
    Analizando ésta fórmula y el caso. La demanda cae, por ende el valor de tener el activo físico no es 
    tan alto (ya que puedo comprarlo en cualquier momento o no tiene plusvalía), es decir $y \rightarrow 0$, 
    aumentando el valor del forward.


  \end{frame}
  
  \section{Pregunta 5}
  
% #region Variables pregunta 5
\newcommand{\tcincounomeses}{6} 
\newcommand{\tcincouno}{\frac{6}{12}} 
\pgfmathsetmacro{\tcincounoc}{6/12}
\newcommand{\ttres}{\frac{3}{12}} 
\pgfmathsetmacro{\ttresc}{3/12}
\newcommand{\litros}{1060000}
\newcommand{\barril}{65}
\newcommand{\galon}{159}
\newcommand{\scerobarril}{48.6}
\newcommand{\almyenv}{0.5}
\newcommand{\tlrdos}{0.03}
\pgfmathsetmacro{\fcincouno}{\almyenv*exp(-\tlrdos*\ttresc)}
\pgfmathsetmacro{\fcincodos}{\almyenv*exp(-\tlrdos*\tcincounoc)}
\pgfmathsetmacro{\fcinco}{\fcincouno+\fcincodos}
\pgfmathsetmacro{\fcincotres}{\scerobarril+\fcinco}
\pgfmathsetmacro{\exxpuno}{\tlrdos*\tcincounoc}
\pgfmathsetmacro{\fcincocuatro}{\fcincotres*exp(\exxpuno)}
\newcommand{\cobmin}{\ensuremath{h^*=\rho\cdot \frac{\sigma_S}{\sigma_F}}}
\newcommand{\devsemp}{1.231}
\newcommand{\devsemf}{1.285}
\newcommand{\coefcorr}{0.94}
\pgfmathsetmacro{\hcincobe}{\coefcorr*\devsemp/\devsemf}
\newcommand{\contrato}{150}
\newcommand{\optimocontratos}{\ensuremath{N^*= h^* \cdot\frac{ Q_A}{Q_F}}}
\newcommand{\qa}{6666.67}
\pgfmathsetmacro{\qentreq}{\qa/\contrato}
\pgfmathsetmacro{\nfinal}{\hcincobe*\qentreq}
\newcommand{\petroleo}{57.5}
\newcommand{\combustible}{75.5}
\newcommand{\ganancia}{{\(\text{Ganacia}  =\left(S_0-K\right)\cdot N^* \cdot \text{tamaño contrato}\)}}
\pgfmathsetmacro{\petrouno}{\petroleo-\fcincocuatro}
\pgfmathsetmacro{\contrauno}{\contrato*40}
\newcommand{\petroleodos}{41.5}
\newcommand{\combustibledos}{58.1}
\pgfmathsetmacro{\petrodos}{\petroleodos-\fcincocuatro}
\pgfmathsetmacro{\petroleodif}{\petroleo-\petroleodos}
\pgfmathsetmacro{\combustibledif}{\combustible-\combustibledos}
\pgfmathsetmacro{\contrados}{\contrato*40}

% #endregion

\begin{frame}{Pregunta 5}
  \justify
  Suponga que LAN está preocupada por las variaciones del precio del petróleo durante los siguientes meses,
  específicamente los siguientes \textbf{\tcincounomeses\ meses}, ya que el combustible que ocupa la flota de aviones ocupa un importante 
  ítem de costo en el presupuesto de la firma, LAN sabe que necesitara \miles{\litros} litros de combustible en \textbf{\tcincounomeses\ meses}
  más para sus operaciones nacionales.\\
  El actual precio de combustible para aviones está en USD \barril\ el barril (\galon\ litros). \\
  El gran problema que posee LAN es que el precio del combustible depende directamente del precio del Barril de Petróleo 
  en el mercado internacional el cual es actualmente de USD \scerobarril\ y solo existen derivados sobre petróleo y no sobre combustible.
  
  \end{frame}

\subsection{Parte a)}

\begin{frame}{Pregunta 5 parte a}
  \textbf{a)} Calcule el precio teórico de un futuro sobre petróleo a 6 meses si existe un costo trimestral de almacenaje y 
  envió de USD \almyenv\ y la tasa libre de riesgo relevante es de \porcentaje{\tlrdos} anual (compuesta continua).
  
\end{frame}

\begin{frame}{Desarrollo  a)}
  Datos: \(S_0=\scerobarril\), \(r=\tlrdos\), \(T=\tcincouno\).\\
  Utilizamos la fórmula \(\forwardalmacenamiento\) y además
  \(U=Alm_1 + Alm_2 \)\\
  \pausa
  \(U= \almyenv \cdot e^{-\tlrdos \cdot \ttres }+\almyenv \cdot e^{-\tlrdos \cdot \tcincouno }\)\\
  \pausa
  \(U= \decimalx{\fcincouno}+\decimalx{\fcincodos}\)\\
  \pausa
  \cajaverde{
  \(U= \decimalx{\fcinco}\)
  }
\end{frame}

\begin{frame}{Desarrollo a) (continuación)}
  Datos: \(S_0=\scerobarril\), \(r=\tlrdos\), \(T=\tcincouno\), \(U=\decimalx{\fcinco}\)\\
  \pausa
  \(F_0= \left(\scerobarril+\decimalx{\fcinco}\right)\cdot e^{\tlrdos \cdot \tcincouno }\)\\
  \pausa
 \(F_0= \decimalx{\fcincotres}\cdot e^{\decimalx{\exxpuno}}\)\\
 \pausa
 \cajaverde{
 \(F_0=\decimalx{\fcincocuatro}\)
 }
\end{frame}

\subsection{Parte b)}
\begin{frame}{Pregunta 5 parte b}
\textbf{b)} Suponga que la desviación estándar de los cambios semestrales en los precios del combustible es USD \decimalx{\devsemp} y 
que la desviación estándar de los cambios semestrales en el precio del futuro del petróleo es de USD \decimalx{\devsemf}, y el coeficiente 
de correlación entre estos cambios es de \coefcorr\ Calcule el ratio de cobertura (hedge ratio) óptimo para un contrato a 6 meses 
y explique cómo LAN podría ocupar el número calculado.
\end{frame}
 
\begin{frame}{Desarrollo b)}
  \Large
  Datos: \(\sigma_s= \decimalx{\devsemp}\),
   \(\sigma_f= \decimalx{\devsemf}\),
   \(\rho= \coefcorr\).\\
  Usando la fórmula:\\
  {\textcolor{blue}{\(\cobmin\)}}\\
  \pausa
  \(h^*=\coefcorr \cdot \frac{\decimalx{\devsemp}}{\decimalx{\devsemf}} \)\\
  \pausa
  \cajaverde{
  \(h^*= \decimalx{\hcincobe}\)
  }\\
  \vspace{.5em}
  El valor de $h^*$ indica el porcentaje de cobertura de los \miles{\litros} litros a comprar en los próximos 6 meses. 
\end{frame}

\subsection{Parte c)}

\begin{frame}{Pregunta 5 parte c}
\Large
  \textbf{c)} Calcule cuántos contratos de petróleo necesitara LAN para la cobertura de 
combustible si cada contrato es de \contrato\ barriles.
\end{frame}

\begin{frame}{Desarrollo c)}
  \large
  Datos: \(h^*=\decimalx{\hcincobe}\),
  \(Q_A=\frac{\miles{\litros}}{\galon }\) \(\rightarrow Q_A=\qa\)
  \(Q_F=\contrato\).\\
  Usando la fórmula: \\
  \textcolor{blue}{\optimocontratos}\\
  \pausa
  \vspace{.5em}
  \pausa
  \(N^*=\decimalx{\hcincobe} \cdot \frac{\entero{\qa}}{\contrato}\)\\
  \pausa
  \(N^*=\decimalx{\hcincobe} \cdot \decimalx{\qentreq}\)\\
  \pausa
  \(N^*=\decimalx{\nfinal} \)\\
  \pausa
  Aproximamos\\
  \cajaverde{
    \(N^*=\entero{\nfinal} \)
}
\end{frame}

\subsection{Parte d)}
\begin{frame}{Pregunta 5 parte d}
  \textbf{d)}  Cuál sería el resultado de la estrategia de LAN si en 6 meses más el precio del petróleo fuera de USD \petroleo\ el barril y el de combustible USD \combustible.
\end{frame}

\begin{frame}{Desarrollo d)}
  Datos:
  \(S_0= \petroleo\),
  \(K= \decimalx{\fcincocuatro}\),
  \(N^*=\entero{\nfinal}\), 
  \(\text{tamaño de contratos} =\contrato\).\\
  Utilizamos la fórmula: \textcolor{blue}{\ganancia}
  \(Ganancia= (\petroleo-\decimalx{\fcincocuatro})\cdot \entero{\nfinal}\cdot \contrato\)\\
  \pausa
  \(Ganacia= \decimalx{\petrouno} \cdot \entero{\contrauno}\)\\
  \pausa
  \cajaverde{
    \(Ganacia= \dinero{42972}\)
    }
  \end{frame}

  \begin{frame}{Desarrollo d) (continuación)}
  Lan debe comprar obligatoriamente los litros de combustible, por ende \\

  Datos: combustible en galones= \(\entero{\qa}\), valor del galón de combustible= \(\combustible\)  \\
  \pausa
  \(Compra=\entero{\qa}\cdot \combustible\)\\
  \pausa
  \cajaverde{
    \(Compra= \dinero{503334}\) 
    }
    \pausa
\end{frame}
    
\begin{frame}{Desarrollo d) (continuación)}
  \Large
  \(Ingreso\ neto= Ganancia-Compra\) \\
  \pausa
  \(Ingreso\ neto= \dinero{42972} -\dinero{503334} \) \\
  \pausa
  \cajaverde{
    \(Ingreso\ neto=  -\dinero{460361} \) 
    }

\end{frame}
  
\subsection{Parte e)}
\begin{frame}{Pregunta 5 parte e}
  \textbf{e)} Cuál sería el resultado de la estrategia de LAN si en 6 meses más el precio del petróleo fuera de USD \petroleodos\ el barril y el de combustible de USD \combustibledos\ .
\end{frame}
  
  
\begin{frame}{Desarrollo e)}
  Datos:
  \(S_0= \petroleodos\),
  \(K= \decimalx{\fcincocuatro}\),
  \(N^*=\entero{\nfinal}\), 
  \(\text{tamaño de contratos} =\contrato\).\\
  Utilizamos la fórmula: \textcolor{blue}{\ganancia}
  \(Ganancia= (\petroleodos-\decimalx{\fcincocuatro})\cdot \entero{\nfinal}\cdot \contrato\)\\
  \pausa
  \(Ganacia= \decimalx{\petrodos} \cdot \entero{\contrauno}\)\\
  \pausa
  \cajaverde{
  \(Ganacia= \dinero{-53028}\)
  }
\end{frame}

\begin{frame}{Desarrollo e) (continuación)}
  Lan debe comprar obligatoriamente los litros de combustible, por ende \\

  Datos: combustible en galones= \(\entero{\qa}\), valor del galón de combustible= \(\combustibledos\) \\
  \pausa 
  \(Compra=\entero{\qa}\cdot \combustibledos\)\\
  \pausa
  \cajaverde{
 \(Compra= \dinero{387353}\)
    }
    \pausa
\end{frame}

\begin{frame}{Desarrollo e) (continuación)}
  \Large
  \(Ingreso\ neto= Ganancia-Compra\) \\
  \pausa
  \(Ingreso\ neto= \dinero{-53028} -\dinero{387353} \) \\
  \pausa
  \cajaverde{
  \(Ingreso\ neto=  -\dinero{-440381  } \) 
  }
\end{frame}

\begin{frame}{comparacíon d) y e)}
  \begin{table}[h!]
    \centering
    \scriptsize
    \caption{Balance de LAN}
    \begin{tabular}{|c|c|c|c|c|}
    \hline
    \textbf{ejercicio}&\textbf{petroleo}& \textbf{combustible} & \textbf{ IN con contrato} & \textbf{IN sin contrato} \\
    \hline
     d)&\dinero{\petroleo}&\dinero{\combustible}&-\dinero{460361}&-\dinero{503334}\\
     e)&\dinero{\petroleodos}&\dinero{\combustibledos}&-\dinero{440381}&-\dinero{387353}\\
     \hline
     Diferencia&\dinero{\petroleodif}&\dinero{\combustibledif}&\dinero{19980}&\dinero{115981}\\

    \hline
    \end{tabular}
  \end{table}  

\end{frame}

\section{Pregunta 6}
%  #region variables pregunta 6


% #endregion

\begin{frame}{Pregunta 6}
  El gerente de un fondo de inversión en acciones nacionales mantiene una cartera valorada en 18 millones de US\$ con un Beta de 
  1,25 en septiembre del 2022; el gerente está preocupado sobre el comportamiento del mercado durante los próximos once meses y 
  piensa en utilizar contratos de futuro a doce meses sobre el índice IPSA para cubrir el riesgo. El nivel del índice el 01/09/2022 
  es de 4.055 y un contrato de futuros es sobre 250 veces el índice. La tasa de interés libre de riesgo es de 3\% anual, la tasa de 
  rendimiento por dividendos sobre las acciones del índice es de 0.8\% anual y el tipo de cambio de dicho día fue de \$800  Se pide:
  \end{frame}
  
  \begin{frame}{Pregunta 6}
    \begin{enumerate}[label=\textbf{\alph*)}]
      \item Cuál es el precio teórico del futuro para el contrato de futuros de 12 meses.
      \item Qué posición en contratos de futuros debe tomar el gerente del Fondo, para realizar una cobertura óptima a lo largo de los próximos 12 meses.
      \item Calcule el efecto de su estrategia sobre el rendimiento del Fondo que administra el gerente, si el nivel del mercado en septiembre del 2023 del índice es 4.144.
      \item Calcule el efecto de su estrategia sobre el rendimiento del Fondo que administra el gerente, si el nivel del mercado en septiembre del 2023 del índice es 3.615.
      
    \end{enumerate}
\end{frame} 
  
\subsection{Parte a)}

\begin{frame}{Pregunta 6 parte a)}
  \textbf{a)} Cuál es el precio teórico del futuro para el contrato de futuros de 12 meses.
\end{frame}
  \begin{frame}{desarrolo a)}
    Datos: \(S_0=\miles{4055}\), \(r=0.03\), \(q=0.008\), \(T=1\)\\
    Utilizamos la fórmula \textcolor{blue}{\(\forwarddividendoper\)}\\
    \pausa
  \(F_0 =  \miles{4055} \cdot{e^{\left(0.03-0.008\right)\cdot 1}}\)\\
  \pausa
  \(F_0 =  \miles{4055} \cdot{e^{0.022}}\)\\
  \pgfmathsetmacro{\eveintidos}{exp(.022)}
  \pausa
  \(F_0 =  \miles{4055} \cdot\decimalx{\eveintidos}\)\\
  \pgfmathsetmacro{\seisa}{4055*\eveintidos}
  \pausa
  \cajaverde{
    \(F_0 = \miles{\seisa} \)
    }
  \end{frame}
  \subsection{Parte b)}
  \begin{frame}{Pregunta 6 parte b)}
    \textbf{b)} Qué posición en contratos de futuros debe tomar el gerente del Fondo, para realizar una cobertura óptima a lo largo de los próximos 12 meses.
  \end{frame}
  \begin{frame}{Desarrollo b)}
    Datos: valor a invertir=USD \(\miles{18000000}\),
    tasa de cambio= \(800 CLP/USD\),
    \(F_0=\miles{4155}\),
    tamaño del contrato = \(250 * IPSA\), Beta= \(1.25\)
    \\
    \textcolor{blue}{\(N^*=\beta \cdot \frac{V_A}{V_F}\)}\\
    \pausa
    \(N^*=\beta \cdot \frac{V_A}{\text{Tamaño contrato }\cdot F_0}\)\\
    \pausa
    \(N^*=1.25 \cdot \frac{\miles{18000000 }\cdot 800}{250\cdot \miles{4145}}\)\\
    \pausa
    \(N^*=1.25 \cdot \frac{\miles{14400000000}}{\miles{60314}}\)\\
    \pausa
    \(N^*=1.25 \cdot \miles{13896}\)\\
    \pausa
    \cajaverde{
      \(N^*=\miles{17370}\)
      }
  \end{frame}
  \begin{comment}
  \subsection{Parte c)}
  \begin{frame}{Pregunta 6 parte c)}
  \textbf{c)} Calcule el efecto de su estrategia sobre el rendimiento del Fondo que administra el gerente, si el nivel del mercado en septiembre del 2023 del índice es 4.144.
  \end{frame}

  \subsection{Parte d)}
  \begin{frame}{Pregunta 6 parte d)}
  \textbf{d)} Calcule el efecto de su estrategia sobre el rendimiento del Fondo que administra el gerente, si el nivel del mercado en septiembre del 2023 del índice es 3.615.
  \end{frame}
  
\end{comment}

\end{document}