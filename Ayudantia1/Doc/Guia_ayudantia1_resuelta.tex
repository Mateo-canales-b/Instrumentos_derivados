
\documentclass[12pt]{article}
\usepackage[utf8]{inputenc}
\usepackage{amsmath}
\usepackage{graphicx}
\usepackage{xcolor}
\usepackage{geometry}
\usepackage{enumitem}
\usepackage{fancyhdr}
\usepackage{titling}
\geometry{letterpaper, margin=1in}
\fancypagestyle{firststyle}{
    \fancyhf{}
    \lhead{\includegraphics[height=5cm]{../imagenes/logo.png}}
    \renewcommand{\headrulewidth}{0pt}
}
\pagestyle{plain}

\definecolor{rojoudp}{RGB}{210,35,42}
\newcommand{\subrayadoRojo}[1]{{\color{rojoudp}\underline{\textcolor{black}{#1}}}}
\newcommand{\formulasection}[2]{%
    \vspace{1em}
    \noindent
    {\large\bfseries\color{blue}#1}\\[-4em]
    \begin{center}
        \large
      \[
      #2
      \]
    \end{center}
    \vspace{1em}
}
\title{Ayudantía Nº1: Curva Cero}
\author{Curso: Instrumentos Derivados\\
Profesor: Francisco Rantul\\
Ayudante: Camila Luardo}
\date{31/03/2025}

\setlength{\droptitle}{-3em}
\begin{document}
\begin{figure}
    \vspace{-5em}    
    \flushright
    \includegraphics[height=4cm]{../imagenes/logo.png}\\[-3em]
\end{figure}
\begin{center}
    {\LARGE \textbf{Ayudantía Nº1: Curva Cero}}\\[0.5em]
    Curso: Instrumentos Derivados\\
    Profesor: Francisco Rantul\\
    Ayudante: Mateo Canales\\
    \date{31/03/2025}
\end{center}
\vspace{1pt}
{\color{rojoudp}\hrule height 2pt}
\vspace{10pt}

Los precios de los pagarés descontables del Banco Central de Chile (PDBC)
 a 6 meses y a 1 año son de \$94 y \$89 respectivamente los cuales pagan \$100. 
  
 Un bono del Banco Central de Chile en pesos (BCP) a 1,5 años que paga cupón de 
 \$4 cada 6 meses tiene un precio de \$94{,}84. Un BCP a 2 años que paga
 cupón de \$5 cada 6 meses tiene un precio de \$97{,}12.

\section*{\subrayadoRojo{Preguntas}}

\begin{enumerate}[label=\textbf{\alph*)}]
    \item Calcule la curva cero de 6 meses, 1 año, 1,5 años y 2 años.
     Utilice capitalización continua.
    
    \item Grafique la curva cero y comente (sin realizar cálculos) si 
    la pendiente de la curva de los bonos del BCCh (con cupones) es 
    positiva o negativa. ¿Qué factor explica el \textit{spread} entre 
    ambas curvas?, ¿Por qué el \textit{spread} aumenta a mayor madurez?
    
    \item Comente cuál es la interpretación económica detrás de la 
    pendiente observada en la curva cero. ¿Qué nos dice respecto a 
    la probabilidad de recesión?
    
    \item Considerando que usted tiene la información de la curva cero, 
    la curva \textit{forward} y la curva de las \textit{yields} de los 
    bonos de gobierno. Señale qué curva usaría para calcular el valor 
    presente de las ganancias o pérdidas de los contratos \textit{forward}.
    
    \item ¿Cuál es el rol de las probabilidades neutrales al riesgo en d)?, 
    ¿Qué rol juega la condición de no arbitraje?
    
    \item Calcule el punto a) utilizando matrices en Excel/R/Phyton.
\end{enumerate}

\newpage

\section*{\subrayadoRojo{Respuestas}}

\subsection*{a) Cálculo de la Curva Cero (capitalización continua)}

\formulasection{Fórmula de Interés Continuo}{
    F = P \cdot e^{-rt}
}
\vspace{-3em}
\begin{quote}
    \scriptsize
Donde:
\begin{itemize}
    \item $F$ es el valor futuro del instrumento o flujo.
    \item $P$ es el valor presente (monto actual o precio del instrumento).
    \item $r$ es la tasa de interés continua expresada en forma decimal.
    \item $t$ es el tiempo al vencimiento en años.
    \item $e$ es la base de los logaritmos naturales, aproximadamente igual a 2.71828.
\end{itemize}
\end{quote}

\paragraph{PDBC a 6 meses:}
\begin{align*}
100 e^{-r \cdot \frac{6}{12}} &= 94 \\
e^{-r \cdot 0.5} &= \frac{94}{100} \\
-r \cdot 0.5 &= \ln\left(\frac{94}{100}\right) \\
r &= -\frac{\ln(0.94)}{0.5} = 0.1238 = 12{,}38\%
\end{align*}

\paragraph{PDBC a 1 año:}
\begin{align*}
100 e^{-r \cdot 1} &= 89 \\
e^{-r} &= \frac{89}{100} \\
r &= -\ln\left(\frac{89}{100}\right) = 0.1165 = 11{,}65\%
\end{align*}

\paragraph{Bono a 1{,}5 años:}
\begin{align*}
4 e^{-0.1238 \cdot 0.5} + 4 e^{-0.1165 \cdot 1} + 104 e^{-r \cdot 1.5} &= 94.84 \\
3.76 + 3.56 + 104 e^{-r \cdot 1.5} &= 94.84 \\
104 e^{-r \cdot 1.5} &= 94.84 - 3.76 - 3.56 = 87.52 \\
e^{-r \cdot 1.5} &= \frac{87.52}{104} = 0.8415 \\
r &= -\frac{\ln(0.8415)}{1.5} = 0.115 = 11{,}5\%
\end{align*}

\paragraph{Bono a 2 años:}
\begin{align*}
5 e^{-0.1238 \cdot 0.5} + 5 e^{-0.1165 \cdot 1} + 5 e^{-0.115 \cdot 1.5} + 105 e^{-r \cdot 2} &= 97.12 \\
4.70 + 4.45 + 4.21 + 105 e^{-r \cdot 2} &= 97.12 \\
105 e^{-r \cdot 2} &= 97.12 - 4.70 - 4.45 - 4.21 = 83.76 \\
e^{-r \cdot 2} &= \frac{83.76}{105} = 0.7977 \\
r &= -\frac{\ln(0.7977)}{2} = 0.113 = 11{,}3\%
\end{align*}

\newpage

\subsection*{b) Gráfico y comentario de la pendiente}

\begin{quote}
Como la pendiente de la curva cero es negativa, la pendiente de la curva de bonos del BCCh (con cupones) también es negativa (ambas curvas siempre tienen la misma pendiente). El factor que explica el \textit{spread} entre ambas curvas es el cupón. A mayor madurez, el \textit{spread} aumenta porque los cupones van tomando mayor relevancia en el valor presente total de los flujos.
\end{quote}

\subsection*{c) Interpretación económica de la pendiente}

\begin{quote}
Si la pendiente de la curva es negativa, se espera un escenario de bajo crecimiento económico, por lo tanto, las tasas de interés serán más bajas que las observadas actualmente. De forma indirecta, se espera que el BCCh buscará incentivar el consumo disminuyendo la TPM. Esto sugiere una mayor probabilidad de recesión.
\end{quote}

\subsection*{d) Curva para valorar contratos forward}

\begin{quote}
Para calcular el valor presente de las ganancias o pérdidas de los contratos \textit{forward}, se debe utilizar la curva cero, es decir, las tasas de los bonos de gobierno sin el efecto de los cupones (\textit{bootstrapping}).
\end{quote}

\subsection*{e) Probabilidades neutrales al riesgo y no arbitraje}

\begin{quote}
En una economía con probabilidades neutrales al riesgo, el retorno esperado de los instrumentos financieros es la tasa libre de riesgo. Por lo tanto, los instrumentos se valorizan descontando los flujos a la tasa libre de riesgo, eliminando el efecto de los cupones (curva cero). Para que existan probabilidades neutrales al riesgo no debe existir arbitraje.
\end{quote}

\[
\frac{S_{\min}}{S} < 1 + r < \frac{S_{\max}}{S}
\]

\subsection*{f) Reproducción en Excel/R/Phyton}

\begin{quote}
Replicar los cálculos del punto a) utilizando matrices en Excel, R o en Phyton.
\end{quote}

\end{document}