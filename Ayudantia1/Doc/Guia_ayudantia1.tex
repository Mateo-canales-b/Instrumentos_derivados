\documentclass[12pt]{article}
\usepackage[utf8]{inputenc}
\usepackage{amsmath}
\usepackage{graphicx}
\usepackage{xcolor}
\usepackage{geometry}
\usepackage{enumitem}
\usepackage{fancyhdr}
\usepackage{titling}
\geometry{letterpaper, margin=1in}
\fancypagestyle{firststyle}{
    \fancyhf{}
    \lhead{\includegraphics[height=5cm]{../imagenes/logo.png}}
    \renewcommand{\headrulewidth}{0pt}
}
\pagestyle{plain}

\definecolor{rojoudp}{RGB}{210,35,42}
\newcommand{\subrayadoRojo}[1]{{\color{rojoudp}\underline{\textcolor{black}{#1}}}}
\newcommand{\formulasection}[2]{%
    \vspace{1em}
    \noindent
    {\large\bfseries\color{blue}#1}\\[-4em]
    \begin{center}
        \large
      \[
      #2
      \]
    \end{center}
    \vspace{1em}
}
\title{Ayudantía Nº1: Curva Cero}
\author{Curso: Instrumentos Derivados\\
Profesor: Francisco Rantul\\
Ayudante: Camila Luardo}
\date{31/03/2025}

\setlength{\droptitle}{-3em}
\begin{document}
\begin{figure}
    \vspace{-5em}    
    \flushright
    \includegraphics[height=4cm]{../imagenes/logo.png}\\[-3em]
\end{figure}
\begin{center}
    {\LARGE \textbf{Ayudantía Nº1: Curva Cero}}\\[0.5em]
    Curso: Instrumentos Derivados\\
    Profesor: Francisco Rantul\\
    Ayudante: Mateo Canales\\
    \date{31/03/2025}
\end{center}
\vspace{1pt}
{\color{rojoudp}\hrule height 2pt}
\vspace{10pt}

Los precios de los pagarés descontables del Banco Central de Chile (PDBC)
 a 6 meses y a 1 año son de \$94 y \$89 respectivamente, los cuales pagan \$100. 
  
 Un bono del Banco Central de Chile en pesos (BCP) a 1,5 años que paga cupón de 
 \$4 cada 6 meses tiene un precio de \$94{,}84. Un BCP a 2 años que paga
 cupón de \$5 cada 6 meses tiene un precio de \$97{,}12.

\section*{\subrayadoRojo{Preguntas}}

\begin{enumerate}[label=\textbf{\alph*)}]
    \item Calcule la curva cero de 6 meses, 1 año, 1,5 años y 2 años.
     Utilice capitalización continua.
    
    \item Grafique la curva cero y comente (sin realizar cálculos) si 
    la pendiente de la curva de los bonos del BCCh (con cupones) es 
    positiva o negativa. ¿Qué factor explica el \textit{spread} entre 
    ambas curvas?, ¿Por qué el \textit{spread} aumenta a mayor madurez?
    
    \item Comente cuál es la interpretación económica detrás de la 
    pendiente observada en la curva cero. ¿Qué nos dice respecto a 
    la probabilidad de recesión?
    
    \item Considerando que usted tiene la información de la curva cero, 
    la curva \textit{forward} y la curva de las \textit{yields} de los 
    bonos de gobierno. Señale qué curva usaría para calcular el valor 
    presente de las ganancias o pérdidas de los contratos \textit{forward}.
    
    \item ¿Cuál es el rol de las probabilidades neutrales al riesgo en d)?, 
    ¿Qué rol juega la condición de no arbitraje?
    
    \item Calcule el punto a) utilizando matrices en Excel/R/Phyton.
\end{enumerate}
\end{document}