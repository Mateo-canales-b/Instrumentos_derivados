\documentclass{beamer}
\usetheme{Darmstadt}
\usepackage{comment}
\usepackage[utf8]{inputenc}    % Para escribir acentos y caracteres especiales
\usepackage{graphicx}          % Para insertar imágenes
\usepackage{booktabs}          % Para tablas más bonitas
\usepackage{enumitem}
\usepackage{amsmath}           % Para escribir ecuaciones
\usepackage{tikz}              % Para gráficos vectoriales
\usetikzlibrary{calc}          % Para coordenadas calculadas con TikZ
\usepackage{pgfplots}          % Para gráficos matemáticos
\pgfplotsset{compat=1.18}      % Evita errores por compatibilidad

\newif\ifpresentacion
%\presentaciontrue  % Cambiar a \presentacionfalse para PDF sin animaciones
\presentacionfalse % Cambiar a \presentaciontrue para PDF con  animaciones

\usepackage{ifthen}
\newcommand{\pausa}{\ifpresentacion\pause\fi}

\title{Ayudantía 1 \\ Bonos \\ \large\textit{Instrumentos Derivados}}
\author{
  \texorpdfstring{
    \textbf{Profesor:} Francisco Rantul \\[0.3em]
    \textbf{Ayudante:} Mateo Canales
  }{Profesor: Francisco Rantul, Ayudante: Mateo Canales}
}
\subject{Instrumentos Derivados}
\institute{Universidad Diego Portales}
\date{31 De Marzo, 2025}

\begin{document}

% Portada
\begin{frame}
    \titlepage
    \vfill
    \centering
    \includegraphics[width=2.3118cm]{../imagenes/logo.png}
  \end{frame}
% Caso
\section{Caso}
  \begin{frame}
    
  \frametitle{Caso}

  Los precios de los pagarés descontables del Banco Central de Chile (PDBC)
  a 6 meses y a 1 año son de \$94 y \$89 respectivamente. 
  
  Un bono del Banco Central de Chile en pesos (BCP) a 1,5 años que paga cupón de 
  \$4 cada 6 meses tiene un precio de \$94{,}84. Un BCP a 2 años que paga
  cupón de \$5 cada 6 meses tiene un precio de \$97{,}12.

  \end{frame}
  % Contenido
\begin{frame}
  \frametitle{Contenido}
  \tableofcontents
\end{frame}
  % Pregunta a)
\section{Pregunta \text{a)}}
  \begin{frame}
    \frametitle{Preguntas}
    \begin{itemize}
      \footnotesize
      \item {\Large\textcolor{blue}{a) Calcule la curva cero de 6 meses, 1 año, 1,5 años y 2 años. Utilice capitalización continua.}}
      \vspace{3pt}
      \item {\textcolor{white}{b) Grafique la curva cero y comente (sin realizar cálculos) si la pendiente de la curva de los bonos del BCCh (con cupones) es positiva o negativa. ¿Qué factor explica el \textit{spread} entre ambas curvas?, ¿por qué el \textit{spread} aumenta a mayor madurez?}}
      \vspace{3pt}
      \item {\textcolor{white}{c) Comente cuál es la interpretación económica detrás de la pendiente observada en la curva cero. ¿Qué nos dice respecto a la probabilidad de recesión?}}
      \vspace{3pt}
      \item {\textcolor{white}{d) Considerando que usted tiene la información de la curva cero, la curva \textit{forward} y la curva de las \textit{yields} de los bonos de gobierno. Señale qué curva usaría para calcular el valor presente de las ganancias o pérdidas de los contratos \textit{forward}.}}
      \vspace{3pt}
      \item {\textcolor{white}{e) ¿Cuál es el rol de las probabilidades neutrales al riesgo en d)?, ¿qué rol juega la condición de no arbitraje?}}
      \vspace{3pt}
      \item {\textcolor{white}{f) Calcule el punto a) utilizando matrices en Excel/R/Phyton.}}
      \vspace{3pt}
    \end{itemize}
  \end{frame}
\subsection{Parte 1}
% Pregunta a parte 1
\begin{frame}
  \frametitle{Pregunta \text{a)} parte 1}
  \LARGE \underbar{Calcule la curva cero de 6 meses} \\[1em]
  
  \footnotesize
  \textbf{Dato:} Precio del PDBC a 6 meses = \$94
  
  \pausa
  
  \textbf{Fórmula (capitalización continua):}
  \[
    F = P \cdot e^{-rt}
    \]
    
    \pausa
    
    Como $F = 100$, entonces:
    \[
      100 = 94 \cdot e^{-r \cdot 0.5}
      \]
      
      \pausa
      
      \textbf{Despejando $r$:}
      \[
        r = -\frac{1}{0.5} \cdot \ln\left(\frac{100}{94}\right) \approx 0.126
        \]
        
        \pausa
        
        \textbf{Resultado:} la tasa cero a 6 meses es aproximadamente 12.6\%
      \end{frame}
\subsection{Parte 2}
% Pregunta a parte 2
  \begin{frame}
    \frametitle{Pregunta \text{a)} parte 2}
    \LARGE \underbar{Calcule la curva cero de 1 año} \\[1em]
    
    \footnotesize
    \textbf{Dato:} Precio del PDBC a 1 año = \$89
    
    \pausa
    
  \textbf{Fórmula (capitalización continua):}
  \[
  F = P \cdot e^{-rt}
  \]

  \pausa

  Como $F = 100$, entonces:
  \[
  100 = 89 \cdot e^{-r \cdot 1}
  \]

  \pausa

  \textbf{Despejando $r$:}
  \[
  r = -\ln\left(\frac{100}{89}\right) \approx 0.117
  \]

  \pausa

  \textbf{Resultado:} la tasa cero a 1 año es aproximadamente 11.7\%
  \end{frame}

\section{Pregunta \text{b)}}
% Pregunta b)
\begin{frame}
  \frametitle{Preguntas}
  \begin{itemize}
    \footnotesize
    \item {\textcolor{black}{a) Calcule la curva cero de 6 meses, 1 año, 1,5 años y 2 años. Utilice capitalización continua.}}
    \vspace{3pt}
    \item {\Large\textcolor{blue}{b) Grafique la curva cero y comente (sin realizar cálculos) si la pendiente de la curva de los bonos del BCCh (con cupones) es positiva o negativa. ¿Qué factor explica el \textit{spread} entre ambas curvas?, ¿por qué el \textit{spread} aumenta a mayor madurez?}}
    \vspace{3pt}
    \item {\textcolor{white}{c) Comente cuál es la interpretación económica detrás de la pendiente observada en la curva cero. ¿Qué nos dice respecto a la probabilidad de recesión?}}
    \vspace{3pt}
    \item {\textcolor{white}{d) Considerando que usted tiene la información de la curva cero, la curva \textit{forward} y la curva de las \textit{yields} de los bonos de gobierno. Señale qué curva usaría para calcular el valor presente de las ganancias o pérdidas de los contratos \textit{forward}.}}
    \vspace{3pt}
    \item {\textcolor{white}{e) ¿Cuál es el rol de las probabilidades neutrales al riesgo en d)?, ¿qué rol juega la condición de no arbitraje?}}
    \vspace{3pt}
    \item {\textcolor{white}{f) Calcule el punto a) utilizando matrices en Excel/R/Phyton.}}
    \vspace{3pt}
  \end{itemize}
\end{frame}
\section{Pregunta \text{c)}}
% Pregunta c)
\begin{frame}
  \frametitle{Preguntas}
  \begin{itemize}
    \footnotesize
    \item {\textcolor{black}{a) Calcule la curva cero de 6 meses, 1 año, 1,5 años y 2 años. Utilice capitalización continua.}}
    \vspace{3pt}
    \item {\textcolor{black}{b) Grafique la curva cero y comente (sin realizar cálculos) si la pendiente de la curva de los bonos del BCCh (con cupones) es positiva o negativa. ¿Qué factor explica el \textit{spread} entre ambas curvas?, ¿por qué el \textit{spread} aumenta a mayor madurez?}}
    \vspace{3pt}
    \item {\Large\textcolor{blue}{c) Comente cuál es la interpretación económica detrás de la pendiente observada en la curva cero. ¿Qué nos dice respecto a la probabilidad de recesión?}}
    \vspace{3pt}
    \item {\textcolor{white}{d) Considerando que usted tiene la información de la curva cero, la curva \textit{forward} y la curva de las \textit{yields} de los bonos de gobierno. Señale qué curva usaría para calcular el valor presente de las ganancias o pérdidas de los contratos \textit{forward}.}}
    \vspace{3pt}
    \item {\textcolor{white}{e) ¿Cuál es el rol de las probabilidades neutrales al riesgo en d)?, ¿qué rol juega la condición de no arbitraje?}}
    \vspace{3pt}
    \item {\textcolor{white}{f) Calcule el punto a) utilizando matrices en Excel/R/Phyton.}}
    \vspace{3pt}
  \end{itemize}
\end{frame}

\section{Pregunta \text{d)}}
% Pregunta d)
\begin{frame}
  \frametitle{Preguntas}
  \begin{itemize}
    \footnotesize
    \item {\textcolor{black}{a) Calcule la curva cero de 6 meses, 1 año, 1,5 años y 2 años. Utilice capitalización continua.}}
    \vspace{3pt}
    \item {\textcolor{black}{b) Grafique la curva cero y comente (sin realizar cálculos) si la pendiente de la curva de los bonos del BCCh (con cupones) es positiva o negativa. ¿Qué factor explica el \textit{spread} entre ambas curvas?, ¿por qué el \textit{spread} aumenta a mayor madurez?}}
    \vspace{3pt}
    \item {\textcolor{black}{c) Comente cuál es la interpretación económica detrás de la pendiente observada en la curva cero. ¿Qué nos dice respecto a la probabilidad de recesión?}}
    \vspace{3pt}
    \item {\Large\textcolor{blue}{d) Considerando que usted tiene la información de la curva cero, la curva \textit{forward} y la curva de las \textit{yields} de los bonos de gobierno. Señale qué curva usaría para calcular el valor presente de las ganancias o pérdidas de los contratos \textit{forward}.}}
    \vspace{3pt}
    \item {\textcolor{white}{e) ¿Cuál es el rol de las probabilidades neutrales al riesgo en d)?, ¿qué rol juega la condición de no arbitraje?}}
    \vspace{3pt}
    \item {\textcolor{white}{f) Calcule el punto a) utilizando matrices en Excel/R/Phyton.}}
    \vspace{3pt}
  \end{itemize}
\end{frame}

\section{Pregunta \text{e)}}
% Pregunta e)
\begin{frame}
  \frametitle{Preguntas}
  \begin{itemize}
    \footnotesize
    \item {\textcolor{black}{a) Calcule la curva cero de 6 meses, 1 año, 1,5 años y 2 años. Utilice capitalización continua.}}
    \vspace{3pt}
    \item {\textcolor{black}{b) Grafique la curva cero y comente (sin realizar cálculos) si la pendiente de la curva de los bonos del BCCh (con cupones) es positiva o negativa. ¿Qué factor explica el \textit{spread} entre ambas curvas?, ¿por qué el \textit{spread} aumenta a mayor madurez?}}
    \vspace{3pt}
    \item {\textcolor{black}{c) Comente cuál es la interpretación económica detrás de la pendiente observada en la curva cero. ¿Qué nos dice respecto a la probabilidad de recesión?}}
    \vspace{3pt}
    \item {\textcolor{black}{d) Considerando que usted tiene la información de la curva cero, la curva \textit{forward} y la curva de las \textit{yields} de los bonos de gobierno. Señale qué curva usaría para calcular el valor presente de las ganancias o pérdidas de los contratos \textit{forward}.}}
    \vspace{3pt}
    \item {\Large\textcolor{blue}{e) ¿Cuál es el rol de las probabilidades neutrales al riesgo en d)?, ¿qué rol juega la condición de no arbitraje?}}
    \vspace{3pt}
    \item {\textcolor{white}{f) Calcule el punto a) utilizando matrices en Excel/R/Phyton.}}
    \vspace{3pt}
  \end{itemize}
\end{frame}
\section{Pregunta \text{f)}}
% Pregunta f)
\begin{frame}
  \frametitle{Preguntas}
  \begin{itemize}
    \footnotesize
    \item {\textcolor{black}{a) Calcule la curva cero de 6 meses, 1 año, 1,5 años y 2 años. Utilice capitalización continua.}}
    \vspace{3pt}
    \item {\textcolor{black}{b) Grafique la curva cero y comente (sin realizar cálculos) si la pendiente de la curva de los bonos del BCCh (con cupones) es positiva o negativa. ¿Qué factor explica el \textit{spread} entre ambas curvas?, ¿por qué el \textit{spread} aumenta a mayor madurez?}}
    \vspace{3pt}
    \item {\textcolor{black}{c) Comente cuál es la interpretación económica detrás de la pendiente observada en la curva cero. ¿Qué nos dice respecto a la probabilidad de recesión?}}
    \vspace{3pt}
    \item {\textcolor{black}{d) Considerando que usted tiene la información de la curva cero, la curva \textit{forward} y la curva de las \textit{yields} de los bonos de gobierno. Señale qué curva usaría para calcular el valor presente de las ganancias o pérdidas de los contratos \textit{forward}.}}
    \vspace{3pt}
    \item {\textcolor{black}{e) ¿Cuál es el rol de las probabilidades neutrales al riesgo en d)?, ¿qué rol juega la condición de no arbitraje?}}
    \vspace{3pt}
    \item {\Large\textcolor{blue}{f) Calcule el punto a) utilizando matrices en Excel/R/Phyton.}}
    \vspace{3pt}
  \end{itemize}
\end{frame}
\end{document}