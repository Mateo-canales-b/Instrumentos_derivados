\documentclass{beamer}
\usetheme{Darmstadt}
\usepackage{comment}
\usepackage[utf8]{inputenc}    % Para escribir acentos y caracteres especiales
\usepackage{graphicx}          % Para insertar imágenes
\usepackage{booktabs}          % Para tablas más bonitas
\usepackage{enumitem}
\usepackage{amsmath}           % Para escribir ecuaciones
\usepackage{tikz}              % Para gráficos vectoriales
\usetikzlibrary{calc}          % Para coordenadas calculadas con TikZ
\usepackage{pgfplots}          % Para gráficos matemáticos
\pgfplotsset{compat=1.18}      % Evita errores por compatibilidad
\usepackage{hyperref}
\usepackage{pgf}
\newif\ifpresentacion
\presentaciontrue  % Cambiar a \presentacionfalse para PDF sin animaciones
%\presentacionfalse % Cambiar a \presentaciontrue para PDF con  animaciones

\usepackage{ifthen}
\newcommand{\pausa}{\ifpresentacion\pause\fi}
% #region Título
\title{Ayudantía 1 \\ Bonos \\ \large\textit{Instrumentos Derivados}}
\author{
  \texorpdfstring{
    \textbf{Profesor:} Francisco Rantul \\[0.3em]
    \textbf{Ayudante:} Mateo Canales
  }{Profesor: Francisco Rantul, Ayudante: Mateo Canales}
}
\subject{Instrumentos Derivados}
\institute{Universidad Diego Portales}
\date{02 De Abril, 2025}
% #endregion 
\begin{document}

% Portada
\begin{frame}
    \titlepage
    \vfill
    \centering
    \includegraphics[width=2.3118cm]{../imagenes/logo.png}
  \end{frame}
%
% Precios de instrumentos
% #region Variables de instrumentos financieros
\newcommand{\Nominal}{100}    % Precio actual para PDBC 6 meses
\newcommand{\Fseis}{94}     % Valor nominal recibido en 6 meses
\newcommand{\Funo}{89}      % Valor nominal recibido en 1 año

\newcommand{\PunoCinco}{94.84}  % Precio BCP 1.5 años
\newcommand{\Pdos}{97.12}       % Precio BCP 2 años

% Flujos de cupones
\newcommand{\CunoCinco}{4}  % Cupón semestral BCP 1.5 años
\newcommand{\Cdos}{5}       % Cupón semestral BCP 2 años

% Tiempos
\newcommand{\Tseis}{0.5}
\newcommand{\Tuno}{1.0}
\newcommand{\TunoCinco}{1.5}
\newcommand{\Tdos}{2.0}

% Flujo final común en bonos con cupón
\newcommand{\Ffinal}{104}
\newcommand{\FfinalDos}{105}
% Tasas cero en formato porcentaje (para mostrar en texto)
\newcommand{\taunio}{11.7\%}
\newcommand{\taaunoCinco}{11.2\%}
\newcommand{\tados}{10.9\%}

% Tasas cero en formato decimal (para usar en fórmulas)
\newcommand{\rtaseis}{0.124}
\newcommand{\rtaunio}{0.117}
\newcommand{\rtaaunoCinco}{0.112}
\newcommand{\rtados}{0.109}

% #endregion
% Caso
\section{Caso}
  \begin{frame}
    
  \frametitle{Caso}

  Los precios de los pagarés descontables del Banco Central de Chile (PDBC)\footnote{\textit{Para 
  más información sobre el PDBC, consulte la página 21 del documento del Banco Central de Chile:}}
  a 6 meses y a 1 año son de \$\Fseis\ y \$\Funo\, respectivamente, los cuales pagan \$\Nominal.
  \vspace{0.5em}
  
  
  Un bono del Banco Central de Chile en pesos (BCP) \footnote{\textit{Para más información sobre el BCP, 
  consulte la página 45 del documento del Banco Central de Chile:}\\
  \tiny{\href{https://www.bcentral.cl/contenido/-/detalle/caracteristicas-de-los-instrumentos-del-mercado-financiero-nacional-3}
  {\texttt{www.bcentral.cl - Características de los instrumentos}}}}
  a 1,5 años que paga cupón de  \$\CunoCinco\  cada 6 meses tiene un precio de \$\PunoCinco. 
  Un BCP a 2 años que paga
  cupón de \$\Cdos\ cada 6 meses tiene un precio de \$\Pdos.

  \end{frame}
  % Contenido
\begin{frame}
    \frametitle{Contenido}
    \tableofcontents
\end{frame}
  % Pregunta a)
\section{Pregunta \text{a)}}
% Pregunta a)
  \begin{frame}
    \frametitle{Preguntas}
    \begin{itemize}
      \scriptsize
      \item {\Large\textcolor{blue}{a) Calcule la curva cero de 6 meses, 1 año, 1,5 años y 2 años. Utilice capitalización continua.}}
      \vspace{3pt}
      \item {\textcolor{white}{b) Grafique la curva cero y comente (sin realizar cálculos) si la pendiente de la curva de los bonos del BCCh (con cupones) es positiva o negativa. ¿Qué factor explica el \textit{spread} entre ambas curvas?, ¿por qué el \textit{spread} aumenta a mayor madurez?}}
      \vspace{3pt}
      \item {\textcolor{white}{c) Comente cuál es la interpretación económica detrás de la pendiente observada en la curva cero. ¿Qué nos dice respecto a la probabilidad de recesión?}}
      \vspace{3pt}
      \item {\textcolor{white}{d) Considerando que usted tiene la información de la curva cero, la curva \textit{forward} y la curva de las \textit{yields} de los bonos de gobierno. Señale qué curva usaría para calcular el valor presente de las ganancias o pérdidas de los contratos \textit{forward}.}}
      \vspace{3pt}
      \item {\textcolor{white}{e) ¿Cuál es el rol de las probabilidades neutrales al riesgo en d)?, ¿qué rol juega la condición de no arbitraje?}}
      \vspace{3pt}
      \item {\textcolor{white}{f) Calcule el punto a) utilizando matrices en Excel/R/Python.}}
      \vspace{3pt}
    \end{itemize}
  \end{frame}

% #region variable pa1
\newcommand{\entero}[1]{\pgfmathprintnumber[fixed, precision=0]{#1}}
\newcommand{\decimal}[1]{\pgfmathprintnumber[fixed, precision=2]{#1}}
\newcommand{\decimalx}[1]{\pgfmathprintnumber[fixed, precision=3]{#1}}
\newcommand{\porcentaje}[1]{%
  \pgfmathsetmacro{\temp}{#1*100}%
  \pgfmathprintnumber[fixed, precision=2]{\temp}\%%
}\newcommand{\capcontinuacero}{ F = P \cdot e^{-rT}}
\newcommand{\capcontinua}{F = \sum_{i=1}^{n} \frac{C_i}{e^{r_i t_i}} + \frac{P}{e^{r_n t_n}}}
\pgfmathsetmacro{\pai}{-(1/\Tseis)*ln(\Fseis/\Nominal)}
% #endregion
\subsection{Parte 1}
% Pregunta a parte 1

\begin{frame}
  \frametitle{Pregunta \text{a)} parte 1}
  \LARGE \underbar{Calcule la curva cero de 6 meses} \\[1em]
  \footnotesize
  \textbf{Datos:} Valor nominal (P) = \$\Nominal, Precio actual (F) = \$\Fseis, Tiempo final (T)=\Tseis. \\
  \pausa 
  \textbf{Fórmula capitalización continua:}
  \[\capcontinua\]
  \pausa
  Dado C=0
  \[\capcontinuacero\]
  \pausa
  Reemplazamos con los datos:
  \[\Fseis = \Nominal \cdot e^{-r \cdot \Tseis}\] \pausa
  \textbf{Despejando $r$:}
  \[r = -\frac{1}{\Tseis} \cdot \ln\left(\frac{\Fseis}{\Nominal}\right) \pausa \approx \decimalx{\pai}  \]
  \pausa
  \textbf{Resultado:} La tasa cero a 6 meses es aproximadamente \porcentaje{\pai}
\end{frame}

\subsection{Parte 2}
\pgfmathsetmacro{\paii}{-(1/\Tuno)*ln(\Funo/\Nominal)}
% Pregunta a parte 2
  \begin{frame}
    \frametitle{Pregunta \text{a)} parte 2}
    \LARGE \underbar{Calcule la curva cero de 1 año} \\[1em]
    \footnotesize
    \textbf{Datos:} Valor nominal (F) = \$\Nominal, Precio actual (P) = \$\Funo,Tiempo final (T)=\Tuno.\\
    \pausa
    \textbf{Fórmula capitalización continua:}
    \[\capcontinua\]
    \pausa
    Dado C=0
    \[\capcontinuacero\]
    \pausa
    Reemplazamos con los datos:
    \pausa
    \textbf{Despejando $r$:}
   \[r = -\ln\left(\frac{\Funo}{\Nominal}\right) \pausa \approx \decimalx{\paii}\]
    \pausa
   \textbf{Resultado:} La tasa cero a 1 año es aproximadamente \porcentaje{\paii}
  \end{frame}

\subsection{Parte 3}

% #region Variables pa3
\pgfmathsetmacro{\fluno}{\CunoCinco* exp(-\pai*\Tseis)}
\pgfmathsetmacro{\fldos}{\CunoCinco* exp(-\paii*\Tuno)}
\pgfmathsetmacro{\flplus}{\fluno + \fldos}
\pgfmathsetmacro{\flmenos}{\PunoCinco - \flplus}
\pgfmathsetmacro{\paiii}{-(1/\TunoCinco)*ln(\flmenos/\Ffinal)}
% #endregion

% Pregunta a parte 3
\begin{frame}
  \frametitle{Pregunta \text{a)} parte 3}
  \LARGE \underbar{Calcule la curva cero de 1.5 años} \\[1em]
  \footnotesize
  \textbf{Datos:} Valor nominal (P) = \$\Nominal, Precio del BCP a 1.5 años = \$\PunoCinco, con cupón de \$\CunoCinco\ cada 6 meses, 
  Tiempo Final = \TunoCinco.\\
  \pausa
  \textbf{Fórmula capitalización continua:}
  \[\capcontinua\]
  \pausa
  \textbf{Cálculo:}
  \[ \PunoCinco = \CunoCinco \cdot e^{\decimalx{-\pai} \cdot \Tseis} + \CunoCinco \cdot e^{\decimalx{-\paii} \cdot \Tuno} + \Ffinal \cdot e^{-r \cdot \TunoCinco}\]
  \pausa
  \textbf{Despejando $r$:}
  \[
  r = -\frac{1}{\TunoCinco} \cdot \ln\left( \frac{\PunoCinco - \CunoCinco e^{\decimalx{-\pai}\cdot \Tseis} - \CunoCinco e^{\decimalx{-\paii}  \cdot \Tuno}}{\Ffinal} \right)
  \]
\end{frame}

% Pregunta a parte 3 (continuación)
\begin{frame}
  \frametitle{Pregunta \text{a)} parte 3 (continuación)}
  \footnotesize
  \textbf{Despejando $r$:}
  \[
    r = -\frac{1}{\TunoCinco} \cdot \ln\left( \frac{\PunoCinco - \CunoCinco e^{\decimalx{-\pai} \cdot \Tseis} - \CunoCinco e^{\decimalx{-\paii}  \cdot \Tuno}}{\Ffinal} \right)
    \]
  \[
   r=-\frac{1}{\TunoCinco} \cdot \ln\left(\frac{\PunoCinco - \decimalx{\flplus}}{\Ffinal} \right)
   \]
  \pausa
  \[
   r=-\frac{1}{\TunoCinco} \cdot \ln\left(\frac{\decimalx{\flmenos}}{\Ffinal} \right)
   \pausa \approx \decimalx{\paiii}
   \]
   
  \textbf{Resultado:} \( r \approx \rtaaunoCinco \) La tasa cero a 1.5 años es aproximadamente \porcentaje{\paiii}
\end{frame}

\subsection{Parte 4}
% #region Variables 4
\pgfmathsetmacro{\fliuno}{\Cdos* exp(-\pai*\Tseis)}
\pgfmathsetmacro{\flidos}{\Cdos* exp(-\paii*\Tuno)}
\pgfmathsetmacro{\flitres}{\Cdos* exp(-\paii*\TunoCinco)}
\pgfmathsetmacro{\fliplus}{\fliuno+\flidos+\flitres}
\pgfmathsetmacro{\flimenos}{\Pdos-\fliplus}
\pgfmathsetmacro{\paiv}{-(1/\Tdos)*ln(\flimenos/\FfinalDos)}

% end region
% Pregunta a parte 4
\begin{frame}
  \frametitle{Pregunta \text{a)} parte 4}
  \LARGE \underbar{Calcule la curva cero de 2 años} \\[1em]

  \footnotesize
  \textbf{Datos:}  Valor nominal (P) = \$\Nominal, Precio del BCP a 2 años = \$\Pdos, 
  con cupón de \$\Cdos\ cada 6 meses, Tiempo Final (T) = \Tdos.
  \pausa
  \textbf{Fórmula capitalización continua:}
  \[\capcontinua\]
  \pausa
  \textbf{Cálculo:}
  \[
  \Pdos = \Cdos \cdot e^{\decimalx{-\pai} \cdot \Tseis} + \Cdos \cdot e^{\decimalx{-\paii}  \cdot \Tuno} + \Cdos \cdot e^{-\paiii \cdot \TunoCinco} + \FfinalDos \cdot e^{-r \cdot \Tdos}
  \]
  \pausa
  \textbf{Despejando $r$:}
  \[
  r = -\frac{1}{\Tdos} \cdot \ln\left( \frac{\Pdos-\Cdos \cdot e^{\decimalx{-\pai} \cdot \Tseis} - \Cdos \cdot e^{\decimalx{-\paii}  \cdot \Tuno} - \Cdos \cdot e^{-\paiii \cdot \TunoCinco}}{\FfinalDos} \right)
  \]

\end{frame}
% Pregunta a parte 4 (continuación)
\begin{frame}
  \frametitle{Pregunta \text{a)} parte 4 (continuación)}
  \footnotesize
  \[
  r = -\frac{1}{\Tdos} \cdot \ln\left( \frac{\Pdos-\Cdos \cdot e^{\decimalx{-\pai} \cdot \Tseis} - \Cdos \cdot e^{\decimalx{-\paii}  \cdot \Tuno} - \Cdos \cdot e^{\decimalx{-\paiii} \cdot \TunoCinco}}{\FfinalDos} \right)
  \]
  \pausa
  \[
  r = -\frac{1}{\Tdos} \cdot \ln\left( \frac{\Pdos-\decimalx{\fliuno} - \decimalx{\flidos} - \decimalx{\flitres}}{\FfinalDos} \right)
  \]
  \pausa
  \[
  r = -\frac{1}{\Tdos} \cdot \ln\left( \frac{\decimalx{\flimenos}}{\FfinalDos} \right)
  \]
  \pausa
  \[
  r = -\frac{1}{\Tdos} \cdot \ln\left( \frac{\decimalx{\flimenos}}{\FfinalDos} \right)
  \pausa \approx \decimal{\paiv}
  \]
  \textbf{Resultado:} \( r \approx \rtados \) La tasa cero a 2 años es aproximadamente \porcentaje{\paiv}
\end{frame}
\section{Pregunta \text{b)}}
% Pregunta b)
\begin{frame}
  \frametitle{Preguntas}
  \begin{itemize}
    \scriptsize
    \item {\textcolor{black}{a) Calcule la curva cero de 6 meses, 1 año, 1,5 años y 2 años. Utilice capitalización continua.}}
    \vspace{3pt}
    \item {\Large\textcolor{blue}{b) Grafique la curva cero y comente (sin realizar cálculos) si la pendiente de la curva de los bonos del BCCh (con cupones) es positiva o negativa. ¿Qué factor explica el \textit{spread} entre ambas curvas?, ¿por qué el \textit{spread} aumenta a mayor madurez?}}
    \vspace{3pt}
    \item {\textcolor{white}{c) Comente cuál es la interpretación económica detrás de la pendiente observada en la curva cero. ¿Qué nos dice respecto a la probabilidad de recesión?}}
    \vspace{3pt}
    \item {\textcolor{white}{d) Considerando que usted tiene la información de la curva cero, la curva \textit{forward} y la curva de las \textit{yields} de los bonos de gobierno. Señale qué curva usaría para calcular el valor presente de las ganancias o pérdidas de los contratos \textit{forward}.}}
    \vspace{3pt}
    \item {\textcolor{white}{e) ¿Cuál es el rol de las probabilidades neutrales al riesgo en d)?, ¿qué rol juega la condición de no arbitraje?}}
    \vspace{3pt}
    \item {\textcolor{white}{f) Calcule el punto a) utilizando matrices en Excel/R/Python.}}
    \vspace{3pt}
  \end{itemize}
\end{frame}
%Grafico
\begin{frame}
  \frametitle{Curva cero (tasa spot)}
  \begin{center}
  \begin{tikzpicture}
    \begin{axis}[
        width=10cm,
        height=6cm,
        xlabel={Plazo (años)},
        ylabel={Tasa cero},
        grid=both,
        ymin=0.08, ymax=0.14,
        yticklabel style={/pgf/number format/.cd, fixed, fixed zerofill, precision=2},
        xtick={0.5,1,1.5,2},
        ytick={0.08,0.10,...,0.14}
      ]
      \addplot+[mark=*] coordinates {
        (\Tseis, 0.124)
        (\Tuno, 0.117)
        (\TunoCinco, 0.112)
        (\Tdos, 0.109)
      };
    \end{axis}
  \end{tikzpicture}
  \end{center}
\end{frame}
\section{Pregunta \text{c)}}
% Pregunta c)
\begin{frame}
  \frametitle{Preguntas}
  \begin{itemize}
    \scriptsize
    \item {\textcolor{black}{a) Calcule la curva cero de 6 meses, 1 año, 1,5 años y 2 años. Utilice capitalización continua.}}
    \vspace{3pt}
    \item {\textcolor{black}{b) Grafique la curva cero y comente (sin realizar cálculos) si la pendiente de la curva de los bonos del BCCh (con cupones) es positiva o negativa. ¿Qué factor explica el \textit{spread} entre ambas curvas?, ¿por qué el \textit{spread} aumenta a mayor madurez?}}
    \vspace{3pt}
    \item {\Large\textcolor{blue}{c) Comente cuál es la interpretación económica detrás de la pendiente observada en la curva cero. ¿Qué nos dice respecto a la probabilidad de recesión?}}
    \vspace{3pt}
    \item {\textcolor{white}{d) Considerando que usted tiene la información de la curva cero, la curva \textit{forward} y la curva de las \textit{yields} de los bonos de gobierno. Señale qué curva usaría para calcular el valor presente de las ganancias o pérdidas de los contratos \textit{forward}.}}
    \vspace{3pt}
    \item {\textcolor{white}{e) ¿Cuál es el rol de las probabilidades neutrales al riesgo en d)?, ¿qué rol juega la condición de no arbitraje?}}
    \vspace{3pt}
    \item {\textcolor{white}{f) Calcule el punto a) utilizando matrices en Excel/R/Python.}}
    \vspace{3pt}
  \end{itemize}
\end{frame}

\section{Pregunta \text{d)}}
% Pregunta d)
\begin{frame}
  \frametitle{Preguntas}
  \begin{itemize}
    \scriptsize
    \item {\textcolor{black}{a) Calcule la curva cero de 6 meses, 1 año, 1,5 años y 2 años. Utilice capitalización continua.}}
    \vspace{3pt}
    \item {\textcolor{black}{b) Grafique la curva cero y comente (sin realizar cálculos) si la pendiente de la curva de los bonos del BCCh (con cupones) es positiva o negativa. ¿Qué factor explica el \textit{spread} entre ambas curvas?, ¿por qué el \textit{spread} aumenta a mayor madurez?}}
    \vspace{3pt}
    \item {\textcolor{black}{c) Comente cuál es la interpretación económica detrás de la pendiente observada en la curva cero. ¿Qué nos dice respecto a la probabilidad de recesión?}}
    \vspace{3pt}
    \item {\Large\textcolor{blue}{d) Considerando que usted tiene la información de la curva cero, la curva \textit{forward} y la curva de las \textit{yields} de los bonos de gobierno. Señale qué curva usaría para calcular el valor presente de las ganancias o pérdidas de los contratos \textit{forward}.}}
    \vspace{3pt}
    \item {\textcolor{white}{e) ¿Cuál es el rol de las probabilidades neutrales al riesgo en d)?, ¿qué rol juega la condición de no arbitraje?}}
    \vspace{3pt}
    \item {\textcolor{white}{f) Calcule el punto a) utilizando matrices en Excel/R/Python.}}
    \vspace{3pt}
  \end{itemize}
\end{frame}

\section{Pregunta \text{e)}}
% Pregunta e)
\begin{frame}
  \frametitle{Preguntas}
  \begin{itemize}
    \scriptsize
    \item {\textcolor{black}{a) Calcule la curva cero de 6 meses, 1 año, 1,5 años y 2 años. Utilice capitalización continua.}}
    \vspace{3pt}
    \item {\textcolor{black}{b) Grafique la curva cero y comente (sin realizar cálculos) si la pendiente de la curva de los bonos del BCCh (con cupones) es positiva o negativa. ¿Qué factor explica el \textit{spread} entre ambas curvas?, ¿por qué el \textit{spread} aumenta a mayor madurez?}}
    \vspace{3pt}
    \item {\textcolor{black}{c) Comente cuál es la interpretación económica detrás de la pendiente observada en la curva cero. ¿Qué nos dice respecto a la probabilidad de recesión?}}
    \vspace{3pt}
    \item {\textcolor{black}{d) Considerando que usted tiene la información de la curva cero, la curva \textit{forward} y la curva de las \textit{yields} de los bonos de gobierno. Señale qué curva usaría para calcular el valor presente de las ganancias o pérdidas de los contratos \textit{forward}.}}
    \vspace{3pt}
    \item {\Large\textcolor{blue}{e) ¿Cuál es el rol de las probabilidades neutrales al riesgo en d)?, ¿qué rol juega la condición de no arbitraje?}}
    \vspace{3pt}
    \item {\textcolor{white}{f) Calcule el punto a) utilizando matrices en Excel/R/Python.}}
    \vspace{3pt}
  \end{itemize}
\end{frame}
\section{Pregunta \text{f)}}
% Pregunta f)
\begin{frame}
  \frametitle{Preguntas}
  \begin{itemize}
    \scriptsize
    \item {\textcolor{black}{a) Calcule la curva cero de 6 meses, 1 año, 1,5 años y 2 años. Utilice capitalización continua.}}
    \vspace{3pt}
    \item {\textcolor{black}{b) Grafique la curva cero y comente (sin realizar cálculos) si la pendiente de la curva de los bonos del BCCh (con cupones) es positiva o negativa. ¿Qué factor explica el \textit{spread} entre ambas curvas?, ¿por qué el \textit{spread} aumenta a mayor madurez?}}
    \vspace{3pt}
    \item {\textcolor{black}{c) Comente cuál es la interpretación económica detrás de la pendiente observada en la curva cero. ¿Qué nos dice respecto a la probabilidad de recesión?}}
    \vspace{3pt}
    \item {\textcolor{black}{d) Considerando que usted tiene la información de la curva cero, la curva \textit{forward} y la curva de las \textit{yields} de los bonos de gobierno. Señale qué curva usaría para calcular el valor presente de las ganancias o pérdidas de los contratos \textit{forward}.}}
    \vspace{3pt}
    \item {\textcolor{black}{e) ¿Cuál es el rol de las probabilidades neutrales al riesgo en d)?, ¿qué rol juega la condición de no arbitraje?}}
    \vspace{3pt}
    \item {\Large\textcolor{blue}{f) Calcule el punto a) utilizando matrices en Excel/R/Python.}}
    \vspace{3pt}
  \end{itemize}
\end{frame}
\end{document}