\documentclass{beamer}
\usetheme{Darmstadt}
\usepackage{comment}
\usepackage[utf8]{inputenc}    % Para escribir acentos y caracteres especiales
\usepackage{graphicx}          % Para insertar imágenes
\usepackage{booktabs}          % Para tablas más bonitas
\usepackage{enumitem}
\usepackage{amsmath}           % Para escribir ecuaciones
\usepackage{tikz}              % Para gráficos vectoriales
\usetikzlibrary{calc}          % Para coordenadas calculadas con TikZ
\usepackage{pgfplots}          % Para gráficos matemáticos
\pgfplotsset{compat=1.18}      % Evita errores por compatibilidad

\title{Ayudantía 1 \\ Bonos \\ \large\textit{Instrumentos Derivados}}
\author{
  \texorpdfstring{
    \textbf{Profesor:} Francisco Rantul \\[0.3em]
    \textbf{Ayudante:} Mateo Canales
  }{Profesor: Francisco Rantul, Ayudante: Mateo Canales}
}
\subject{Instrumentos Derivados}
\institute{Universidad Diego Portales}
\date{31 De Marzo, 2025}

\begin{document}

% Portada
\begin{frame}
    \titlepage
    \vfill
    \centering
    \includegraphics[width=2.3118cm]{../imagenes/logo.png}
  \end{frame}
  \begin{frame}
    \frametitle{Contenido}
    \tableofcontents
  \end{frame}
  \begin{frame}
\section{Caso}
  \frametitle{Caso}

  Los precios de los pagarés descontables del Banco Central de Chile (PDBC)
  a 6 meses y a 1 año son de \$94 y \$89 respectivamente. 
  
  Un bono del Banco Central de Chile en pesos (BCP) a 1,5 años que paga cupón de 
  \$4 cada 6 meses tiene un precio de \$94{,}84. Un BCP a 2 años que paga
  cupón de \$5 cada 6 meses tiene un precio de \$97{,}12.

\end{frame}
\section{Pregunta \text{a)}}
  \begin{frame} 
  
\begin{enumerate}[label=\textbf{\alph*)}]
  \footnotesize
  \item Calcule la curva cero de 6 meses, 1 año, 1,5 años y 2 años.
   Utilice capitalización continua.
   \vspace{4pt}
   
   \item Grafique la curva cero y comente (sin realizar cálculos) si 
   la pendiente de la curva de los bonos del BCCh (con cupones) es 
   positiva o negativa. ¿Qué factor explica el \textit{spread} entre 
   ambas curvas?, ¿por qué el \textit{spread} aumenta a mayor madurez?
   \vspace{4pt}
   
   \item Comente cuál es la interpretación económica detrás de la 
   pendiente observada en la curva cero. ¿Qué nos dice respecto a 
   la probabilidad de recesión?
   
   \vspace{4pt}
   \item Considerando que usted tiene la información de la curva cero, 
   la curva \textit{forward} y la curva de las \textit{yields} de los 
   bonos de gobierno. Señale qué curva usaría para calcular el valor 
   presente de las ganancias o pérdidas de los contratos \textit{forward}.
   \vspace{4pt}
   
   \item ¿Cuál es el rol de las probabilidades neutrales al riesgo en d)?, 
   ¿qué rol juega la condición de no arbitraje?
   \vspace{4pt}
  
  \item Calcule el punto a$)$ utilizando matrices en Excel/R/Phyton.
\end{enumerate}

  \end{frame}  

%Pregunta a parte 1
\begin{frame}
      \frametitle{Pregunta \text{a)} parte 1}
      \LARGE \underbar{Calcule la curva cero de 6 meses} \\[1em]
  
      \footnotesize
      \textbf{Dato:} Precio del PDBC a 6 meses = \$94
  
      \pause
  
      \textbf{Fórmula (capitalización continua):}
      \[
      F = P \cdot e^{-rt}
      \]
  
      \pause
  
      Como $F = 100$, entonces:
      \[
      100 = 94 \cdot e^{-r \cdot 0.5}
      \]
  
      \pause
  
      \textbf{Despejando $r$:}
      \[
      r = -\frac{1}{0.5} \cdot \ln\left(\frac{100}{94}\right) \approx 0.126
      \]
  
      \pause
  
      \textbf{Resultado:} la tasa cero a 6 meses es aproximadamente 12.6\%
\end{frame}
%Pregunta a parte 2
\begin{frame}
      \frametitle{Pregunta \text{a)} parte 2}
      \LARGE \underbar{Calcule la curva cero de 1 año} \\[1em]
  
      \footnotesize
      \textbf{Dato:} Precio del PDBC a 1 año = \$89
  
      \pause
  
      \textbf{Fórmula (capitalización continua):}
      \[
      F = P \cdot e^{-rt}
      \]
  
      \pause
  
      Como $F = 100$, entonces:
      \[
      100 = 89 \cdot e^{-r \cdot 1}
      \]
  
      \pause
  
      \textbf{Despejando $r$:}
      \[
      r = -\ln\left(\frac{100}{89}\right) \approx 0.117
      \]
  
      \pause
  
      \textbf{Resultado:} la tasa cero a 1 año es aproximadamente 11.7gi\%
\end{frame}
\end{document}