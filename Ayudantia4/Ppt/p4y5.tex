\documentclass{beamer}
\usetheme{Darmstadt}
\usepackage[utf8]{inputenc}
\usepackage{graphicx}
\usepackage{booktabs}
\usepackage{enumitem}
\usepackage{amsmath}
\usepackage{tikz}
\usetikzlibrary{arrows.meta, positioning}
\usetikzlibrary{calc}
\usepackage{pgfplots}
\pgfplotsset{compat=1.18}
\usepackage{hyperref}
\usepackage{pgf}
\usepackage{ragged2e}
\usepackage{ifthen}
\newcommand{\pausa}{}

\title{Ayudantía 4 \\ Resolución Pregunta 4 y 5}
\author{
  \texorpdfstring{
    \textbf{Profesor:} Francisco Rantul \\[0.3em]
    \textbf{Ayudante:} Mateo Canales
  }{Profesor: Francisco Rantul, Ayudante: Mateo Canales}
}
\subject{Instrumentos Derivados}
\institute{Universidad Diego Portales}
\date{04 De Junio, 2025}

\begin{document}

% Portada
\begin{frame}
    \titlepage
    \vfill
    \centering
    \includegraphics[width=2.3118cm]{../imagenes/logo.png}
\end{frame}

%%%%%%%%%%%%%%%%%%%%%%%%%%%%%%%%%%%%%%%%%%%%%%%%%%%%%%%%%%%%%%%%
\section{Pregunta 4.a}
\begin{frame}{Pregunta 4.a}
  \justify
  Los saltos temporales de una opción son de 1 mes ($\Delta_t=\frac{1}{12}$), la tasa de interés libre de riesgo 
  local es del 5\% continua anual y la tasa libre de riesgo extranjera es del 8\% continua anual. 
  La volatilidad es del 12\% anual.

  \vspace{1em}
  \textbf{a)} Calcule $u$, $d$ y $p$ cuando se construye un árbol binomial para divisas.
\end{frame}

\begin{frame}{Resolución 4.a}
\justify
\textbf{Datos:} \\
$r=5\%$ , $r_f =8\%$, $\sigma=12\%$, $\Delta t = \frac{1}{12}$ \\

\vspace{0.5em}
\textbf{Fórmulas:}
$u = e^{\sigma \sqrt{\Delta t}}$;$d = e^{-\sigma \sqrt{\Delta t}}$; $p = \frac{e^{(r-r_f) \Delta t} - d}{u - d}$

\vspace{0.5em}
\textbf{Cálculos:} \\
$\sigma\sqrt{\Delta t} = 0.12 \times \sqrt{1/12} \approx 0.03464$ \\

$u = e^{0.03464} \approx 1.0352$ \\
$d = e^{-0.03464} \approx 0.9659$ \\

$e^{(0.05-0.08)/12} = e^{-0.0025} \approx 0.9975$ \\

$p = \dfrac{0.9975 - 0.9659}{1.0352 - 0.9659} \approx \dfrac{0.0316}{0.0693} \approx 0.456$ \\

\vspace{0.5em}
\textbf{Respuesta:} \\
$u \approx 1.0352$ \\
$d \approx 0.9659$ \\
$p \approx 0.456$ \\
\end{frame}

%%%%%%%%%%%%%%%%%%%%%%%%%%%%%%%%%%%%%%%%%%%%%%%%%%%%%%%%%%%%%%%%
\section{Pregunta 5.a}
\begin{frame}{Pregunta 5.a}
  \justify
  Considere una opción call Americana de una divisa. El valor de la divisa hoy es de \$700, 
el strike Price es de \$710, la tasa libre de riesgo local es del 12\% continua anual 
(la tasa libre de riesgo extranjera es del 4\% continua anual), la volatilidad es del 
40\% anual y la madurez del derivado es de 6 meses.

\vspace{1em}
\textbf{a)} Calcule $u$, $d$ y $p$ para un árbol binomial de dos pasos.
\end{frame}

\begin{frame}{Resolución 5.a}
\justify
\textbf{Datos:} \\
$r=12\%$ , $r_f =4\%$, $\sigma=40\%$, $T=0.5$ años, $N=2$, $\Delta t = 0.25$ \\

\textbf{Fórmulas:} \\
$u = e^{\sigma \sqrt{\Delta t}}$; $d = e^{-\sigma \sqrt{\Delta t}}$; $p = \dfrac{e^{(r-r_f)\Delta t} - d}{u - d}$

\textbf{Cálculos:} \\
$\sigma\sqrt{\Delta t} = 0.4 \times \sqrt{0.25} = 0.4 \times 0.5 = 0.2$ \\

$u = e^{0.2} \approx 1.2214$ \\
$d = e^{-0.2} \approx 0.8187$ \\

$e^{(0.12-0.04)\times 0.25} = e^{0.02} \approx 1.0202$ \\

$p = \dfrac{1.0202 - 0.8187}{1.2214 - 0.8187} \approx \dfrac{0.2015}{0.4027} \approx 0.501$ \\

\textbf{Respuesta:} \\
$u \approx 1.2214$ \\
$d \approx 0.8187$ \\
$p \approx 0.501$ \\
\end{frame}

%%%%%%%%%%%%%%%%%%%%%%%%%%%%%%%%%%%%%%%%%%%%%%%%%%%%%%%%%%%%%%%%
\section{Pregunta 5.b}
\begin{frame}{Pregunta 5.b}
  \justify
  Considere una opción call Americana de una divisa. El valor de la divisa hoy es de \$700, 
el strike Price es de \$710, la tasa libre de riesgo local es del 12\% continua anual 
(la tasa libre de riesgo extranjera es del 4\% continua anual), la volatilidad es del 
40\% anual y la madurez del derivado es de 6 meses.

\vspace{1em}
\textbf{b)} Calcule el valor de la opción (usando árbol binomial de dos pasos).
\end{frame}

\begin{frame}{Resolución 5.b (Valores terminales)}
\justify
\textbf{Datos:} $S_0 = 700$, $K = 710$, $u \approx 1.2214$, $d \approx 0.8187$, $p \approx 0.501$, $\Delta t = 0.25$, $r = 0.12$.

\vspace{0.5em}
\textbf{Valores de la divisa en nodos terminales:} \\
$S_{uu} = 700 \times u \times u \approx 700 \times 1.2214^2 \approx 1044.65$ \\
$S_{ud} = 700 \times u \times d \approx 700 \times 1.2214 \times 0.8187 \approx 700$ \\
$S_{dd} = 700 \times d \times d \approx 700 \times 0.8187^2 \approx 468.76$ \\

\vspace{0.5em}
\textbf{Valor opción call en cada nodo:} \\
$f_{uu} = \max(1044.65 - 710, 0) = 334.65$ \\
$f_{ud} = \max(700 - 710, 0) = 0$ \\
$f_{dd} = \max(468.76 - 710, 0) = 0$
\end{frame}

\begin{frame}{Resolución 5.b (Nodos intermedios y valor inicial)}
\justify
\textbf{Factor de descuento:} $e^{-r\Delta t} = e^{-0.03} \approx 0.9704$ \\

\textbf{Valor en nodo intermedio superior:} \\
$f_u = 0.9704 \times [0.501 \times 334.65 + 0.499 \times 0] \approx 0.9704 \times 167.65 \approx 162.72$ \\

\textbf{Valor en nodo intermedio inferior:} \\
$f_d = 0.9704 \times [0.501 \times 0 + 0.499 \times 0] = 0$ \\

\textbf{Valor en el nodo inicial:} \\
$f_0 = 0.9704 \times [0.501 \times 162.72 + 0.499 \times 0] \approx 0.9704 \times 81.52 \approx 79.11$ \\

\vspace{0.5em}
\textbf{Respuesta final:} \\
$f_0 \approx 79.11$
\end{frame}
\end{document}