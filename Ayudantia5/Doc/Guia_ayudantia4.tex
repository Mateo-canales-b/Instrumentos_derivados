\documentclass[12pt]{article}
\usepackage[utf8]{inputenc}
\usepackage{amsmath}
\usepackage{graphicx}
\usepackage{xcolor}
\usepackage{geometry}
\usepackage{enumitem}
\usepackage{fancyhdr}
\usepackage{float}
\renewcommand{\figurename}{Gráfico}  

\geometry{letterpaper, margin=1in}
\fancypagestyle{firststyle}{
    \fancyhf{}
    \lhead{\includegraphics[height=5cm]{../imagenes/logo.png}}
    \renewcommand{\headrulewidth}{0pt}
    }
    \pagestyle{plain}
    
    \definecolor{rojoudp}{RGB}{210,35,42}
\newcommand{\subrayadoRojo}[1]{{\color{rojoudp}\underline{\textcolor{black}{#1}}}}
\renewcommand{\thesection}{Pregunta \arabic{section}}
\setcounter{section}{1}
\newcommand{\pregunta}[1]{%
  \section*{\subrayadoRojo{\thesection  #1}}%
  \stepcounter{section}%
}
\vspace{-3em}
\begin{document}
\begin{figure}
    \vspace{-5em}    
    \flushright
    \includegraphics[height=4cm]{../imagenes/logo.png}\\[-3em]
\end{figure}
\begin{center}
    {\LARGE \textbf{Ayudantía Nº4: Opciones}}\\[0.5em]
    Curso: Instrumentos Derivados\\
    Profesor: Francisco Rantul\\
    Ayudante: Mateo Canales\\
\end{center}
\vspace{1pt}
{\color{rojoudp}\hrule height 2pt}
\vspace{10pt}

 \pregunta{ Hull 15.13,15.14}
Se sabe que el precio actual de la acción es \$50, no paga dividendos, con un precio de y su vencimiento es de 3 meses,
el precio de ejercicio de \$50 la tasa de interés libre de riesgo es de 10\% 
anual, y la volatilidad es de 30\% anual.
\begin{enumerate}[label=\textbf{\alph*)}]
    \item   {Calcule el valor de d1}
    \item   {Calcule el valor de d2}
    \item   {Calcule el valor de la opción put usando la fórmula de Black-Scholes.}
    \item   {¿Qué diferencia hay si se espera un dividendo de \$1,5 en 2 meses}
\end{enumerate}

 \pregunta{ Hull 15.21}
Se sabe que una acción que no paga dividendos, el precio de la 
acción es de \$52, el precio de ejercicio es de \$50, la tasa de interés libre de riesgo es de 12\% anual, la 
volatilidad es de 30\% anual, y el tiempo hasta el vencimiento es de 3 meses?
\begin{enumerate}[label=\textbf{\alph*)}]
    \item   {Calcule el valor de d1}
    \item   {Calcule el valor de d2}
    \item   {Calcule el valor de la opción call europea usando la fórmula de Black-Scholes.}
\end{enumerate}

 \pregunta{ Hull 15.16}
El precio de una acción sigue un movimiento browniano geométrico con un rendimiento esperado de 16\% y una 
volatilidad de 35\%. El precio actual es de \$38.\\
\begin{enumerate}[label=\textbf{\alph*)}]
    \item	 ¿Cuál es la probabilidad de que una opción call europea sobre la acción con un precio de ejercicio 
    de \$40 y vencimiento en 6 meses sea ejercida?
    \item    ¿Cuál es la probabilidad de que una opción put europea sobre la acción con el mismo precio de 
    ejercicio y vencimiento sea ejercida?
\end{enumerate}


 \pregunta{ Hull 15.35}
El precio de una acción es actualmente \$50. Suponga que el rendimiento esperado de la acción es de 18\% y su 
volatilidad es de 30\%. 
\begin{enumerate}[label=\textbf{\alph*)}]
    \item   ¿Cuál es la distribución de probabilidad para el precio de la acción en 2 años?
    \item   Calcule la media de la distribución
    \item   Calcule desviación estándar de la distribución
    \item   Determine el intervalo de confianza del 95\%.
\end{enumerate}     

\pregunta{ Hull 14.20}

Suponga que \( x \) es el rendimiento al vencimiento (yield to maturity) con capitalización continua de un bono
cupón cero que paga \$1 en el tiempo \( T \). Se asume que \( x \) sigue el siguiente proceso estocástico:

\begin{equation}
dx = (a_1 x_0 - x^2) \, dt + s x \, dz
\end{equation}

Dónde \( a \), \( x_0 \) y \( s \) son constantes positivas, y \( dz \) es un proceso de Wiener. ¿Cuál es el proceso seguido por el precio del bono?

\pregunta{ Hull 15.26}
Demuestre que las fórmulas de Black–Scholes–Merton para opciones call y put satisfacen la paridad put–call.

\end{document}