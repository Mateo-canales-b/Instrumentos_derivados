\documentclass{beamer}
\usetheme{Darmstadt}
\usepackage{comment}
\usepackage[utf8]{inputenc}    % Para escribir acentos y caracteres especiales
\usepackage{graphicx}          % Para insertar imágenes
\usepackage{booktabs}          % Para tablas más bonitas
\usepackage{enumitem}
\usepackage{amsmath}           % Para escribir ecuaciones
\usepackage{tikz}              % Para gráficos vectoriales
\usetikzlibrary{arrows.meta, positioning}
\usetikzlibrary{calc}          % Para coordenadas calculadas con TikZ
\usepackage{pgfplots}          % Para gráficos matemáticos
\pgfplotsset{compat=1.18}      % Evita errores por compatibilidad
\usepackage{hyperref}
\usepackage{pgf}
\usepackage{ragged2e}




\newif\ifpresentacion
\ifdefined\versionpresentacion
  \edef\tempver{\versionpresentacion}
  \ifnum\pdfstrcmp{\tempver}{true}=0
    \presentaciontrue
  \else
    \presentacionfalse
  \fi
\else
  \presentacionfalse
\fi

\usepackage{ifthen}
\newcommand{\pausa}{\ifpresentacion\pause\fi}
% #region Título
\title{Ayudantía 5 \\ Opciones
\\ \large\textit{Instrumentos Derivados}}
\author{
  \texorpdfstring{
    \textbf{Profesor:} Francisco Rantul \\[0.3em]
    \textbf{Ayudante:} Mateo Canales
  }{Profesor: Francisco Rantul, Ayudante: Mateo Canales}
}
\subject{Instrumentos Derivados}
\institute{Universidad Diego Portales}
\date{09 De Junio, 2025}
% #endregion 
\begin{document}

% Portada
\begin{frame}
    \titlepage
    \vfill
    \centering
    \includegraphics[width=2.3118cm]{../imagenes/logo.png}
  \end{frame}

% #region Formateo de números 
\newcommand{\cajaverde}[1]{%
  \fcolorbox{blue}{green!20}{%
    { #1}%
  }
}
\newcommand{\cajaverdeletra}[1]{%
  \fcolorbox{blue}{green!20}{%
    \parbox{0.9\linewidth}{\justifying #1}%
  }
}
\newcommand{\formula}[1]{\textcolor{blue}{#1}}
\newcommand{\entero}[1]{\pgfmathprintnumber[fixed, precision=0]{#1}}
\newcommand{\decimal}[1]{\pgfmathprintnumber[fixed, precision=2]{#1}}
\newcommand{\decimalx}[1]{\pgfmathprintnumber[fixed, precision=3]{#1}}
\newcommand{\decimalxx}[1]{\pgfmathprintnumber[fixed, precision=4]{#1}}
\newcommand{\porcentaje}[1]{%
  \pgfmathsetmacro{\temp}{#1*100}%
  \pgfmathprintnumber[fixed, precision=2]{\temp}\%%
  }
\newcommand{\dinero}[1]{%
  \$\,\pgfmathprintnumber[fixed, precision=0]{#1}
  }
\newcommand{\dineros}[1]{%
  \$\,\pgfmathprintnumber[fixed, precision=1]{#1}
  }
\newcommand{\dineross}[1]{%
  \$\,\pgfmathprintnumber[fixed, precision=2]{#1}
  }
\newcommand{\Desarrollo}[1]{Desarrollo Parte {#1})}

% #endregion



% ----------------------------------------------------------------
% Pregunta 1 
% ----------------------------------------------------------------

\section{Pregunta 1}

% #region Variables pregunta 1
\newcommand{\Suno}{50}
\newcommand{\Kuno}{50}
\newcommand{\rdiez}{0.10}
\newcommand{\sigmaPuno}{0.30}
\newcommand{\Tuno}{0.25}
\newcommand{\Divuno}{1.5}
\newcommand{\Tdos}{\frac{2}{12}} % 2 meses en años
\pgfmathsetmacro{\tdos}{2/12} 

% Fórmulas
\newcommand{\callbsm}{$c =  S_0 \cdot N(d_1)- K \cdot e^{-r \cdot (T)} \cdot N(d_2) $}
\newcommand{\putbsm}{$p = K \cdot e^{-r \cdot (T)} \cdot N(-d_2) - S_0 \cdot N(-d_1)$}
\newcommand{\Duno }{$  d_1 = \frac{\ln(S_0 / K) + \left( r + \frac{\sigma^2}{2} \right) \cdot T}{\sigma \cdot \sqrt{T}}$}
\newcommand{\Ddos}{$d_2 = d_1 - \sigma \cdot \sqrt{T}$}
\newcommand{\dividendo}{$S_1 = S_0 - e^{-r \cdot t} \cdot div $}
\newcommand{\vp}{$vp =e^{-r \cdot t} \cdot valor futuro $}

% Cálculos
\pgfmathsetmacro{\auno}{ln(\Suno/\Kuno)}
\pgfmathsetmacro{\ados}{\sigmaPuno^2/2}
\pgfmathsetmacro{\atres}{sqrt(\Tuno)*\sigmaPuno}
\pgfmathsetmacro{\acuatro}{(\rdiez+ \ados)*\Tuno}
\pgfmathsetmacro{\acinco}{\auno+\acuatro }
\pgfmathsetmacro{\aseis}{\acinco /\atres }
\pgfmathsetmacro{\asiete}{\aseis - \atres}
\pgfmathsetmacro{\aocho}{exp(-\rdiez*\Tuno)}

\newcommand{\deuno}{\decimalx{\aseis}}
\newcommand{\dedos}{\decimalx{\asiete}}
\pgfmathsetmacro{\phimenosduno}{0.4040}
\pgfmathsetmacro{\phimenosddos}{0.4633}

\pgfmathsetmacro{\anueve}{\Kuno*\aocho*\phimenosddos}
\pgfmathsetmacro{\adiez}{\Suno*\phimenosduno}
\pgfmathsetmacro{\aonce}{\anueve - \adiez}

\pgfmathsetmacro{\adoce}{exp(-\rdiez*\tdos)}
\pgfmathsetmacro{\atrece}{exp(-\rdiez*\tdos)*\Divuno}
\pgfmathsetmacro{\acatorce}{\Suno-\atrece}
\newcommand{\Sdos}{\decimalx{\acatorce}}
\pgfmathsetmacro{\aquince}{ln(\acatorce/\Kuno)}
\pgfmathsetmacro{\adieciseis}{\aquince+\acinco}
\pgfmathsetmacro{\adiecisiete}{\adieciseis/\atres}
\pgfmathsetmacro{\adieciocho}{\adiecisiete-\atres}
\pgfmathsetmacro{\adiecinueve}{-\adieciocho}

\pgfmathsetmacro{\phidunonuevo}{0.5160}
\pgfmathsetmacro{\phiddosnuevo}{0.4562}

\pgfmathsetmacro{\aveinticuatro}{1-\phiddosnuevo}
\pgfmathsetmacro{\aveintitres}{1-\phidunonuevo}
\pgfmathsetmacro{\aveinte}{\Kuno*\aocho*\aveinticuatro}
\pgfmathsetmacro{\aveintiuno}{\acatorce*\aveintitres}
\pgfmathsetmacro{\aveintidos}{\aveinte-\aveintiuno}




% Peguntas
\newcommand{\Preguno}{Calcule el precio de una opción put europea a 3 meses sobre una acción que no paga dividendos, 
con un precio de ejercicio de $\dinero{\Kuno}$, cuando el precio actual de la acción es $\dinero{\Suno}$, la tasa 
de interés libre de riesgo es de $\porcentaje{\rdiez}$ anual, y la volatilidad es de $\porcentaje{\sigmaPuno}$ anual.
}
\newcommand{\Pregunoa}{Calcule el valor de d1}
\newcommand{\Pregunob}{Calcule el valor de d2}
\newcommand{\Pregunoc}{Calcule el valor de la opción put usando la fórmula de Black-Scholes.}
\newcommand{\Pregunod}{¿Qué diferencia hay si se espera un dividendo de \dinero{\Divuno} en 2 meses}

% #endregion


\begin{frame}
  \frametitle{Pregunta 1}
  \justify
  \Preguno
  \vspace{1em}

\begin{enumerate}[label=\textbf{\alph*)}]
  \item \Pregunoa
  \item \Pregunob
  \item \Pregunoc
  \item \Pregunod
\end{enumerate}

\end{frame}

\subsection{Parte a)}

\begin{frame}{Pregunta 1 Parte a)}
  \justify
  \Preguno
  \vspace{1em}
  
  \textbf{a)}  \Pregunoa
  
\end{frame}

\begin{frame}{\Desarrollo{a}}
\justify
Datos: $S_0 = \dinero{\Suno}$, $K = \dinero{\Kuno}$, $r = \porcentaje{\rdiez}$, $T = \decimalx{\Tuno}$,
 $\sigma = \porcentaje{\sigmaPuno}$\\
\vspace{1em}
Calculamos d1 usando la fórmula:\\
\vspace{.3em}
\textbf{Fórmula:} \formula{\Duno}\\ \pausa
\vspace{.2em}
$  d_1 = \frac{\ln(\Suno / \Kuno) + \left( \rdiez + \frac{\sigmaPuno^2}{2} \right) \cdot \Tuno}{\sigmaPuno\cdot \sqrt{\Tuno}}$\\\pausa
\vspace{.2em}
$  d_1 = \frac{\decimalxx{\auno} + \left( \rdiez + \ados \right) \cdot \Tuno}{\atres}$\\\pausa
\vspace{.1em}
$  d_1 = \frac{ \decimalxx{\acuatro}}{\atres}$\\\pausa
\vspace{.1em}
\cajaverde{$  d_1 = \decimalx{\aseis} $}
\end{frame}

\subsection{Parte b)}

\begin{frame}{Pregunta 1 Parte b)}
  \justify
  \Preguno
  \vspace{1em}
  
  \textbf{b)}  \Pregunob
  
\end{frame}

\begin{frame}{\Desarrollo{b} }
  Datos: $S_0 = \dinero{\Suno}$, $K = \dinero{\Kuno}$, $r = \porcentaje{\rdiez}$, $T = \decimalx{\Tuno}$,
 $\sigma = \porcentaje{\sigmaPuno}$, $  d_1 = \decimalx{\aseis} $ \\

Calculamos d2 usando la fórmula:\\
\vspace{.3em}
\textbf{Fórmula:} \formula{\Ddos}\\ \pausa

$d_2 = \decimalx{\aseis}- \sigmaPuno \cdot \sqrt{\Tuno}$\\ \pausa
$d_2 = \decimalx{\aseis}- \atres$\\ \pausa
\cajaverde{$d_2 = \decimalx{\asiete}$}\\ \pausa
\end{frame}

\subsection{Parte c)}

\begin{frame}{Pregunta 1 Parte c)}
  \justify
  \Preguno
  \vspace{1em}
  
  \textbf{c)}  \Pregunoc
  
\end{frame}

\begin{frame}{\Desarrollo{c}}
    Datos: $S_0 = \dinero{\Suno}$, $K = \dinero{\Kuno}$, $r = \porcentaje{\rdiez}$, $T = \decimalx{\Tuno}$,
 $\sigma = \porcentaje{\sigmaPuno}$, $  d_1 = \deuno $, $d_2 = \dedos$ \\
 \vspace{0.2em}
\textbf{Fórmula:} \formula{\putbsm}\\ \pausa
$p = \Kuno \cdot e^{-\rdiez \cdot \Tuno} \cdot N(-\dedos) - \Suno \cdot N(-\deuno) $\\ \pausa
$p = \Kuno \cdot \decimalxx{\aocho} \cdot \phimenosddos - \Suno \cdot \phimenosduno$ \\ \pausa
$p = \decimalxx{\anueve} - \decimalxx{\adiez}$\\ \pausa
\cajaverde{$p = \decimalxx{\aonce}$}
\end{frame}

\subsection{Parte d)}

\begin{frame}{Pregunta 1 Parte d)}
  \justify
  \Preguno
  \vspace{1em}
  
  \textbf{d)}  \Pregunod
  
\end{frame}

\begin{frame}{\Desarrollo{d}}
  \justify
    Datos: $S_0 = \dinero{\Suno}$, $K = \dinero{\Kuno}$, $r = \porcentaje{\rdiez}$, $T = \decimalx{\Tuno}$,
 $\sigma = \porcentaje{\sigmaPuno}$, $  d_1 = \deuno $, $d_2 = \dedos$, $t=\Tdos$ \\

  \vspace{1em}
Al precio actual de la acción debemos restarle el valor presente de los dividendos.\\
\textbf{Fórmula:} \formula{\vp}\\ \pausa
$vp= e^{-\rdiez \cdot \decimalx{\tdos}} \cdot \Divuno$\\ \pausa
$vp= \decimalx{\adoce}\cdot \Divuno$\\ \pausa
$vp= \decimalx{\atrece}$\\ \pausa
Luego, el nuevo valor de la acción está dado por:\\
$S_1 = \Suno - \decimalx{\atrece} $\\ \pausa
$S_1 =  \decimalx{\acatorce} $\\ \pausa

  \vspace{1em}
  Repetimos los cálculos anteriores con el nuevo valor de la acción:
\end{frame}

\begin{frame}{\Desarrollo{d}}
  \vspace{0.5em}
      Datos: $S_0 = \dineross{\acatorce}$, $K = \dinero{\Kuno}$, $r = \porcentaje{\rdiez}$, $T = \decimalx{\Tuno}$,
 $\sigma = \porcentaje{\sigmaPuno}$, $  d_1 = \deuno $, $d_2 = \dedos$, $t=\Tdos$ \\

  \textbf{Fórmula:} \formula{\Duno}\\ \pausa
  $  d_1 = \frac{\ln(\Sdos / \Kuno) + \left( \rdiez + \frac{\sigmaPuno^2}{2} \right) \cdot \Tuno}{\sigmaPuno\cdot \sqrt{\Tuno}}\pausa=
  \frac{\aquince+\decimalx{\acinco}}{\atres}\pausa
  =\frac{\decimalx{\adieciseis}}{\atres}$\\\pausa
 $ d_1 = \decimalx{\adiecisiete}$\\ \pausa
 $d_2 = \decimalx{\adiecisiete} - \sigmaPuno \cdot \sqrt{\Tuno} \pausa
  = \decimalx{\adiecisiete} - \atres$\\ \pausa
 $d_2 = \decimalx{\adieciocho}$\\ \pausa

  \vspace{1em}
  Ahora, calculamos el valor de la opción put con los nuevos valores:
\end{frame}

\begin{frame}{\Desarrollo{d}}
    Datos: $S_0 = \dineross{\acatorce}$, $K = \dinero{\Kuno}$, $r = \porcentaje{\rdiez}$, $T = \decimalx{\Tuno}$,
 $\sigma = \porcentaje{\sigmaPuno}$, $  d_1 = \decimalx{\adiecisiete} $, $d_2 = \decimalx{\adieciocho}$ \\
 \vspace{0.2em}
\textbf{Fórmula:} \formula{\putbsm}\\ \pausa
$P = \Kuno \cdot e^{-\rdiez \cdot \Tuno} \cdot  N(\decimalx{\adiecinueve})- \decimal{\acatorce} \cdot N(-\decimalx{\adiecisiete})$\\ \pausa
$P = \Kuno \cdot e^{-\rdiez \cdot \Tuno} \cdot  \decimalx{\aveinticuatro}- \decimal{\acatorce} \cdot \aveintitres$\\ \pausa
$P = \decimalx{\aveinte} - \decimalx{\aveintiuno}$\\ \pausa
\cajaverde{$p = \decimalx{\aveintidos}$}\\ \pausa
\vspace{1em}
\textbf{Comparación:}\\
Sin dividendo: \cajaverde{\decimalx{\aonce}}\\
Con dividendo: \cajaverde{\decimalx{\aveintidos}}\\

\textbf{Conclusión:} El precio de la opción put aumenta al considerar el dividendo, 
ya que reduce el valor actual del activo subyacente.
\end{frame}

% ----------------------------------------------------------------
% Pregunta 2
% ----------------------------------------------------------------

\section{Pregunta 2}

% #region Variables pregunta 2
\newcommand{\Stres}{52}

\newcommand{\rdoce}{0.12}



% Fórmulas

% Cálculos
\pgfmathsetmacro{\buno}{ln(\Stres/\Kuno)}
\pgfmathsetmacro{\bdos}{\sigmaPuno^2/2}
\pgfmathsetmacro{\btres}{sqrt(\Tuno)*\sigmaPuno}
\pgfmathsetmacro{\bcuatro}{(\rdoce+ \bdos)*\Tuno}
\pgfmathsetmacro{\bcinco}{\buno+\bcuatro }
\pgfmathsetmacro{\bseis}{\bcinco /\btres }
\pgfmathsetmacro{\bsiete}{\bseis - \btres}
\pgfmathsetmacro{\bocho}{exp(-\rdoce*\Tuno)}

\newcommand{\deunob}{\decimalx{\bseis}}
\newcommand{\dedosb}{\decimalx{\bsiete}}
\pgfmathsetmacro{\phidunob}{0.7032}
\pgfmathsetmacro{\phiddosb}{0.6509}

\pgfmathsetmacro{\bnueve}{\Kuno*\bocho*\phiddosb}
\pgfmathsetmacro{\bdiez}{\Stres*\phidunob}
\pgfmathsetmacro{\bonce}{ \bdiez- \bnueve}



% Peguntas
\newcommand{\Pregdos}{Se sabe que una acción que no paga dividendos, el precio de la 
acción es de $\dinero{\Stres}$, el precio de ejercicio es de $\dinero{\Kuno}$, la tasa de interés libre de riesgo es de $\porcentaje{\rdoce}$, la 
volatilidad es de $\porcentaje{\sigmaPuno}$ anual, y el tiempo hasta el vencimiento es de 3 meses?
  }

% #endregion


\begin{frame}
  \frametitle{Pregunta 2}
  \justify
  \Pregdos
  \vspace{1em}

\begin{enumerate}[label=\textbf{\alph*)}]
  \item \Pregunoa
  \item \Pregunob
  \item \Pregunoc
\end{enumerate}

\end{frame}

\subsection{Parte a)}

\begin{frame}{Pregunta 2 Parte a)}
  \justify
  \Pregdos
  \vspace{1em}
  
  \textbf{a)}  \Pregunoa
  
\end{frame}

\begin{frame}{\Desarrollo{a}}
\justify
Datos: $S_0 = \dinero{\Stres}$, $K = \dinero{\Kuno}$, $r = \porcentaje{\rdoce}$, $T = \decimalx{\Tuno}$,
 $\sigma = \porcentaje{\sigmaPuno}$\\
\vspace{1em}
Calculamos d1 usando la fórmula:\\
\vspace{.3em}
\textbf{Fórmula:} \formula{\Duno}\\ \pausa
\vspace{.2em}
$  d_1 = \frac{\ln(\Stres / \Kuno) + \left( \rdoce + \frac{\sigmaPuno^2}{2} \right) \cdot \Tuno}{\sigmaPuno\cdot \sqrt{\Tuno}}$\\\pausa
\vspace{.2em}
$  d_1 = \frac{\decimalxx{\buno} + \left( \rdoce + \bdos \right) \cdot \Tuno}{\btres}$\\\pausa
\vspace{.1em}
$  d_1 = \frac{ \decimalxx{\bcuatro}}{\btres}$\\\pausa
\vspace{.1em}
$  d_1 = \frac{\decimalxx{\bcinco} }{\btres}$\\\pausa
\vspace{.1em}
\cajaverde{$  d_1 = \decimalx{\bseis} $}
\end{frame}

\subsection{Parte b)}

\begin{frame}{Pregunta 2 Parte b)}
  \justify
  \Pregdos
  \vspace{1em}
  
  \textbf{b)}  \Pregunob
  
\end{frame}

\begin{frame}{\Desarrollo{b} }
  Datos: $S_0 = \dinero{\Stres}$, $K = \dinero{\Kuno}$, $r = \porcentaje{\rdoce}$, $T = \decimalx{\Tuno}$,
 $\sigma = \porcentaje{\sigmaPuno}$, $  d_1 = \decimalx{\bseis} $ \\

Calculamos d2 usando la fórmula:\\
\vspace{.3em}
\textbf{Fórmula:} \formula{\Ddos}\\ \pausa

$d_2 = \decimalx{\bseis}- \sigmaPuno \cdot \sqrt{\Tuno}$\\ \pausa
$d_2 = \decimalx{\bseis}- \btres$\\ \pausa
\cajaverde{$d_2 = \decimalx{\bsiete}$}\\ \pausa
\end{frame}

\subsection{Parte c)}

\begin{frame}{Pregunta 2 Parte c)}
  \justify
  \Pregdos
  \vspace{1em}
  
  \textbf{c)}  \Pregunoc
  
\end{frame}

\begin{frame}{\Desarrollo{c}}
    Datos: $S_0 = \dinero{\Stres}$, $K = \dinero{\Kuno}$, $r = \porcentaje{\rdoce}$, $T = \decimalx{\Tuno}$,
 $\sigma = \porcentaje{\sigmaPuno}$, $  d_1 = \deunob $, $d_2 = \dedosb$ \\
 \vspace{0.2em}
\textbf{Fórmula:} \formula{\callbsm}\\ \pausa
$c = \Stres \cdot N(\deunob) -\Kuno \cdot e^{-\rdoce \cdot \Tuno}\cdot N(\dedosb)$\\ \pausa
$c = \Stres \cdot \phidunob -\Kuno \cdot e^{-\rdoce \cdot \Tuno}\cdot \phiddosb$\\ \pausa
$c = \decimalxx{\bdiez}-\decimalxx{\bnueve} $\\ \pausa
\cajaverde{$c = \decimalxx{\bonce}$}
\end{frame}


% ----------------------------------------------------------------
% Pregunta 3
% ----------------------------------------------------------------

\section{Pregunta 3}

% #region Variables pregunta 3
\newcommand{\rendimiento}{0.16}
\newcommand{\volatilidad}{0.35}
\newcommand{\Scuatro}{38}
\newcommand{\Kcuatro}{40}
\newcommand{\Tcuatro}{0.5} % 6 meses en años


% Fórmulas
\newcommand{\browniano}{$\ln (S_T) \sim \mathcal{N}\left(\ln (S_0) + (\mu - \sigma^2/2)\cdot T,\ \sigma^2 \cdot T\right)$ }
\newcommand{\standarizar}{$Z = \frac{\ln K - \mathbb{E}[\ln S_T]}{\text{desv. estándar}}$}

% Cálculos
\pgfmathsetmacro{\cuno}{ln(\Scuatro)}
\pgfmathsetmacro{\cdos}{\rendimiento-  \volatilidad^2/2} 
\pgfmathsetmacro{\ctres}{\Tcuatro* \volatilidad^2} 
\pgfmathsetmacro{\ccuatro}{\cdos*\Tcuatro} 
\pgfmathsetmacro{\ccinco}{\cuno+\ccuatro} 
\pgfmathsetmacro{\cseis}{ln(\Kcuatro)}
\pgfmathsetmacro{\csiete}{\cseis-\ccinco}
\pgfmathsetmacro{\cocho}{\csiete/\ctres}
\pgfmathsetmacro{\phicuatro}{0.5120}
\pgfmathsetmacro{\cnueve}{1-\phicuatro} 

% Peguntas
\newcommand{\Pregtres}{
  El precio de una acción sigue un movimiento browniano geométrico con un rendimiento esperado de \porcentaje{\rendimiento} y una 
volatilidad de \porcentaje{\volatilidad}. El precio actual es de \dinero{\Scuatro}.
  }
\newcommand{\Pregtresa}{¿Cuál es la probabilidad de que una opción call europea sobre la acción con un precio de ejercicio 
de \dinero{\Kcuatro} y vencimiento en 6 meses sea ejercida?}
\newcommand{\Pregtresb}{¿Cuál es la probabilidad de que una opción put europea sobre la acción conun precio de ejercicio de \dinero{\Kcuatro} y vencimiento en 6 meses sea ejercida?}

% #endregion

\begin{frame}{Pregunta 3}
  \justify
  \Pregtres
  \vspace{1em}

\begin{enumerate}[label=\textbf{\alph*)}]
  \item \Pregtresa
  \item \Pregtresb
\end{enumerate}

\end{frame}

\subsection{Parte a)}

\begin{frame}{Pregunta 3 Parte a)}
  \justify
  \Pregtres
  \vspace{1em}
  
  \textbf{a)}  \Pregtresa
  
\end{frame}

\begin{frame}{\Desarrollo{a}}
  Datos: $S_0 = \dinero{\Scuatro}$, $K = \dinero{\Kcuatro}$, $\mu =\rendimiento$, $T = \decimalx{\Tcuatro}$,
 $\sigma = \porcentaje{\volatilidad}$ \\
 \vspace{0.2em}
\justify
Queremos calcular la probabilidad de que la opción call se ejerza, es decir: 
\[\mathbb{P}(S_T > K)\]
Aplicamos la fórmula de movimiento browniano geométrico:
\formula{\browniano} \pausa

Reemplazamos:\\ \pausa

$\ln (S_T) \sim \mathcal{N} \left(\ln (\Scuatro) + (\rendimiento - \volatilidad^2/2) \cdot \Tcuatro,\ \volatilidad^2 \cdot \Tcuatro \right)$\\ \pausa
$\ln (S_T) \sim \mathcal{N} \left(\decimalx{\cuno} + \decimalx{\cdos} \cdot \Tcuatro,\decimalx{\ctres} \right)$\\ \pausa
$\ln (S_T) \sim \mathcal{N} \left(\decimalx{\cuno} + \decimalx{\ccuatro},\decimalx{\ctres} \right)$\\ \pausa
$\ln (S_T) \sim \mathcal{N} \left(\decimalx{\ccinco},\decimalx{\ctres} \right)$\\ \pausa

\end{frame}

\begin{frame}{\Desarrollo{a}}
  \justify

Dado que \( \ln \Kcuatro = \decimalx{\cseis} \), estandarizamos segun fórmula:\\
\formula{\standarizar}\\ \pausa
$\mathbb{P}(\ln S_T > ln(K_0)) = 1 - N\left(Z\right)$ \\ \pausa
\vspace{.2em}
$\mathbb{P}(\ln S_T > \decimalx{\cseis}) \pausa= 1 - N\left( \frac{\decimalx{\cseis} - \decimalx{\ccinco}}{\decimalx{\ctres}} \right)$ \\ \pausa
\vspace{.2em}
$\mathbb{P}(\ln S_T > \decimalx{\cseis}) = 1 - N\left( \frac{\decimalx{\csiete}}{\decimalx{\ctres}} \right)$ \\ \pausa
\vspace{.2em}
$\mathbb{P}(\ln S_T > \decimalx{\cseis}) = 1 - N\left( \decimalx{\cocho} \right)$ \\ \pausa
$\mathbb{P}(\ln S_T > \decimalx{\cseis}) = 1 - \phicuatro$ \\ \pausa
\cajaverde{$\mathbb{P}(\ln S_T > \decimalx{\cseis}) = \cnueve$} 
\end{frame}

\subsection{Parte b)}


\begin{frame}{\Desarrollo{b}}
\justify
Queremos calcular la probabilidad de que la opción put se ejerza, es decir: 
$\mathbb{P}(S_T < K)$\\ \pausa
Esto equivale a:\\
$\mathbb{P}(\ln S_T < \ln K)$ \\ \pausa
Buscamos:\\
$\mathbb{P}(\ln S_T < \ln K) = N(Z)$\\ \pausa
$\mathbb{P}(\ln S_T < \decimalx{\cseis})= N(\decimalx{\cocho})$\\ \pausa
\cajaverde{$\mathbb{P}(\ln S_T < \decimalx{\cseis}) = \phicuatro$} 

\end{frame}

\end{document}