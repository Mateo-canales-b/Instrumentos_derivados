\documentclass[12pt]{article}
\usepackage[utf8]{inputenc}     % Codificación
\usepackage{amsmath}            % Matemáticas
\usepackage{amssymb}            % Símbolos matemáticos
\usepackage{graphicx}           % Imágenes
\usepackage{tikz}               % Gráficos vectoriales
\usetikzlibrary{arrows.meta, positioning, calc}
\usepackage{pgfplots}           % Gráficos matemáticos
\pgfplotsset{compat=1.18}
\usepackage{xcolor}             % Colores
\usepackage{geometry}           % Márgenes
\usepackage{enumitem}           % Listas personalizadas
\usepackage{fancyhdr}           % Encabezados/pies de página
\usepackage{float}              % Posicionamiento de figuras/tablas
\usepackage{comment}            % Comentarios múltiples
\usepackage{booktabs}           % Tablas profesionales
\usepackage{hyperref}           % Hipervínculos
\usepackage{ragged2e}           % Alineación de texto


\renewcommand{\figurename}{Gráfico}  

\geometry{letterpaper, margin=1in}
\fancypagestyle{firststyle}{
    \fancyhf{}
    \lhead{\includegraphics[height=5cm]{../imagenes/logo.png}}
    \renewcommand{\headrulewidth}{0pt}
    }
    \pagestyle{plain}
    
    \definecolor{rojoudp}{RGB}{210,35,42}
    \definecolor{grene}{RGB}{30,136,61}
\newcommand{\subrayadoRojo}[1]{{\color{rojoudp}\underline{\textcolor{black}{#1}}}}
\renewcommand{\thesection}{Pregunta \arabic{section}}
\setcounter{section}{4}
% Comando para preguntas principales
\newcommand{\pregunta}[1]{%
  \section*{\subrayadoRojo{\thesection  #1}}%
  \stepcounter{section}%
}
% Comando para subpreguntas
\newcounter{subpreg}
\newcommand{\subpregunta}[1]{%
  \subsection*{\subrayadoRojo{#1}}%
}

\begin{document}
\begin{figure}
    \vspace{-5em}    
    \flushright
    \includegraphics[height=4cm]{../imagenes/logo.png}\\[-3em]
\end{figure}
\begin{center}
    {\LARGE \textbf{Formulario opciones}}\\[0.5em]
    Curso: Instrumentos Derivados\\
    Profesor: Francisco Rantul\\
    Ayudante: Mateo Canales\\
\end{center}
\vspace{1pt}
{\color{rojoudp}\hrule height 2pt}
\vspace{10pt}

% #region Fórmulas globales
\newcommand{\arbol}{$p = \frac{e^{r\cdot \Delta t}-d}{u-d}$}
\newcommand{\neutral}{$f = e^{-r \cdot \Delta t}\cdot (p \cdot f_u+(1-p) \cdot f_d)$}
\newcommand{\putcall}{$S_0+p = K \cdot e^{-r \cdot T}+c$}
\newcommand{\ceroud}{$S_0u|d=S_0*(1+(subida|bajada)) $}
\newcommand{\neutrali}{$S_0d\cdot \Delta-f_d=S_0u\cdot \Delta-f_u$}
\newcommand{\portafolio}{$S_0\cdot \Delta-f=(S_0u\cdot \Delta-f_u) \cdot e^{-rT}$}
\newcommand{\calcud}{$u|d=\frac{S_0u|d}{S_0}$}
\newcommand{\callbsm}{$c =  S_0 \cdot \mathcal{N}(d_1)- K \cdot e^{-r \cdot (T)} \cdot \mathcal{N}(d_2) $}
\newcommand{\putbsm}{$p = K \cdot e^{-r \cdot (T)} \cdot \mathcal{N}(-d_2) - S_0 \cdot \mathcal{N}(-d_1)$}
\newcommand{\enemenosuno}{$\mathcal{N}(-d)=1-\mathcal{N}(d)$}
\newcommand{\Duno }{$  d_1 = \frac{\ln(S_0 / K) + \left( r + \frac{\sigma^2}{2} \right) \cdot T}{\sigma \cdot \sqrt{T}}$}
\newcommand{\Ddos}{$d_2 = d_1 - \sigma \cdot \sqrt{T}$}
\newcommand{\dividendo}{$S_1 = S_0 - e^{-r \cdot t} \cdot div $}
\newcommand{\vp}{$vp =e^{-r \cdot t} \cdot valor futuro $}
\newcommand{\browniano}{$\ln (S_T) \sim \mathcal{N}\left(\ln (S_0) + (\mu - \sigma^2/2)\cdot T,\ \sigma^2 \cdot T\right)$ }
\newcommand{\media}{$\mathbb{E}(S_T) = S_0 \cdot e^{\mu \cdot T}$}
\newcommand{\varianza}{$\mathrm{var}(S_T) = S_0^2 \cdot e^{2\cdot \mu \cdot T} \cdot \left( e^{\sigma^2\cdot T} - 1 \right)$}
\newcommand{\standarizar}{$Z = \frac{\ln K - \mathbb{E}[\ln S_T]}{\text{desv.\ estándar}}$}
\newcommand{\intervaloconf}{$IC = \left[ \mathbb{E}(S_T) - Z_{\alpha/2} \cdot \mathrm{sd}{S_T},\ \mathbb{E}(S_T) + Z_{\alpha/2} \cdot \mathrm{sd}(S_T)\right]$}
\newcommand{\ito}{\[df = \left( \frac{\partial f}{\partial t} + \mu \frac{\partial f}{\partial x} + \frac{1}{2} \sigma^2 \frac{\partial^2 f}{\partial x^2} \right) dt + \sigma \frac{\partial f}{\partial x} dz\]
}
% #endregion

% #region Formateo de números 
\newcommand{\cajaverde}[1]{%
  \fcolorbox{blue}{green!20}{%
    { #1}%
  }
}
\newcommand{\cajaverdeletra}[1]{%
  \fcolorbox{blue}{green!20}{%
    \parbox{0.9\linewidth}{\justifying #1}%
  }
}
\newcommand{\formula}[1]{\textcolor{blue}{#1}}
\newcommand{\entero}[1]{\pgfmathprintnumber[fixed, precision=0]{#1}}
\newcommand{\decimal}[1]{\pgfmathprintnumber[fixed, precision=2]{#1}}
\newcommand{\decimalx}[1]{\pgfmathprintnumber[fixed, precision=3]{#1}}
\newcommand{\decimalxx}[1]{\pgfmathprintnumber[fixed, precision=4]{#1}}
\newcommand{\porcentaje}[1]{%
  \pgfmathsetmacro{\temp}{#1*100}%
  \pgfmathprintnumber[fixed, precision=2]{\temp}\%%
}
\newcommand{\dinero}[1]{%
  \$\,\pgfmathprintnumber[fixed, precision=0]{#1}
  }
\newcommand{\dineros}[1]{%
  \$\,\pgfmathprintnumber[fixed, precision=1]{#1}
  }
\newcommand{\dineross}[1]{%
  \$\,\pgfmathprintnumber[fixed, precision=2]{#1}
  }
\newcommand{\Desarrollo}[1]{Desarrollo Parte {#1})}

% #endregion


% #region Variables 

\newcommand{\rendimiento}{0.18}
\newcommand{\volatilidad}{0.30}
\newcommand{\Scuatro}{50}
\newcommand{\Tcuatro}{2} % 2 años
% Variables base utilizadas en el ejercicio Hull 15.35
% Cálculos
\pgfmathsetmacro{\cuno}{ln(\Scuatro)}
\pgfmathsetmacro{\cdos}{\rendimiento-  \volatilidad^2/2} 
\pgfmathsetmacro{\ctres}{\Tcuatro* \volatilidad^2} 
\pgfmathsetmacro{\ccuatro}{\cdos*\Tcuatro} 
\pgfmathsetmacro{\ccinco}{\cuno+\ccuatro} 
\pgfmathsetmacro{\cseis}{\rendimiento *\Tcuatro} 
\pgfmathsetmacro{\csiete}{exp(\cseis)} 
\pgfmathsetmacro{\cocho}{\Scuatro*\csiete} 
\pgfmathsetmacro{\cnueve}{\Scuatro^2} 
\pgfmathsetmacro{\cdiez}{2*\rendimiento*\Tcuatro} 
\pgfmathsetmacro{\conce}{\volatilidad^2*\Tcuatro} 
\pgfmathsetmacro{\cdoce}{exp(\cdiez)} 
\pgfmathsetmacro{\ctrece}{exp(\conce)} 
\pgfmathsetmacro{\ccatorce}{\cnueve*\cdoce} 
\pgfmathsetmacro{\cquince}{\ctrece-1} 
\pgfmathsetmacro{\cdieciseis}{\cquince*\ccatorce} 
\pgfmathsetmacro{\cdiecisiete}{sqrt(\cdieciseis)}
\pgfmathsetmacro{\duno}{1.96*\cdiecisiete} % Z-score para el 95% de confianza
\pgfmathsetmacro{\ddos}{\cocho+\duno}
\pgfmathsetmacro{\dtres}{\cocho-\duno}


% Fórmulas

\newcommand{\Pregcuatro}{El precio de una acción es actualmente \dinero{\Scuatro}. Suponga que el rendimiento esperado de la acción es de~\porcentaje{\rendimiento} y su volatilidad es de \porcentaje{\volatilidad}.}

\newcommand{\Pregcuatroa}{¿Cuál es la distribución de probabilidad para el precio de la acción en \Tcuatro{} años?}
\newcommand{\Pregcuatrob}{Calcule la media de la distribución}
\newcommand{\Pregcuatroc}{Calcule desviación estándar de la distribución}
\newcommand{\Pregcuatrod}{Determine el intervalo de confianza del 95\%.}




% #endregion

\pregunta{ Hull 15.35}
  \justify
  \Pregcuatro
  \vspace{1em}

\begin{enumerate}[label=\textbf{\alph*)}]
  \item \Pregcuatroa
  \item \Pregcuatrob
  \item \Pregcuatroc
  \item \Pregcuatrod
\end{enumerate}


\subpregunta{Parte a)}

Datos: $S_0 = \dinero{\Scuatro}$, $\mu =\rendimiento$, $T = \decimalx{\Tcuatro}$, $\sigma = \porcentaje{\volatilidad}$ \\

\noindent Aplicamos la fórmula de movimiento browniano geométrico: \\[-1em]
\begin{flushleft}
\formula{\browniano} \\[0.5em]
$\ln (S_T) \sim \mathcal{N} \left(\ln (\Scuatro) + (\rendimiento - \volatilidad^2/2) \cdot \Tcuatro,\ \volatilidad^2 \cdot \Tcuatro \right)$\\[0.3em]
$\ln (S_T) \sim \mathcal{N} \left(\decimalx{\cuno} + \decimalx{\cdos} \cdot \Tcuatro,\decimalx{\ctres} \right)$\\[0.3em]
$\ln (S_T) \sim \mathcal{N} \left(\decimalx{\cuno} + \decimalx{\ccuatro},\decimalx{\ctres} \right)$\\[0.3em]
$\ln (S_T) \sim \mathcal{N} \left(\decimalx{\ccinco},\decimalx{\ctres} \right)$\\[0.3em]
\cajaverdeletra{La distibución de Probabilidad del Precio de la acción dado los datos es log normal, ya que los precios no pueden tomar valores negativos, quedando establecido como propiedad.
Con los parámetros de ésta $\mathcal{N} \left(\decimalx{\ccinco},\decimalx{\ctres} \right)$}
\end{flushleft}
\newpage
\subpregunta{Parte b)}
\noindent Según la ecuación del libro (15.4) la media de la distribución se define por:
\begin{flushleft}
\formula{\media}\\[0.5em]
$\mathbb{E}(S_T) = \Scuatro \cdot e^{\rendimiento \cdot \Tcuatro}$\\[0.3em]
$\mathbb{E}(S_T) = \Scuatro \cdot e^{\decimalx{\cseis}}$\\[0.3em]
$\mathbb{E}(S_T) = \Scuatro \cdot e^{\decimalx{\cseis}}$\\[0.3em]
$\mathbb{E}(S_T) = \Scuatro \cdot \decimalx{\csiete}$\\[0.3em]
$\mathbb{E}(S_T) = \decimalx{\cocho}$\\[0.3em]
\cajaverdeletra{Por lo tanto, la media de la distribución es {\dineross{\cocho}}.}
\end{flushleft}

\subpregunta{Parte c)}
\noindent Según la ecuación del libro (15.5) la varianza de la distribución se define por:
\begin{flushleft}
\formula{\varianza}\\ [0.5em]
$\mathrm{var}(S_T) = \Scuatro^2 \cdot e^{2\cdot \rendimiento \cdot \Tcuatro} \cdot \left( e^{\volatilidad^2 \cdot \Tcuatro} - 1 \right)$\\[0.3em]
$\mathrm{var}(S_T) = \entero{\cnueve} \cdot e^{\decimalx{\cdiez}} \cdot \left( e^{\decimalx{\conce}} - 1 \right)$\\[0.3em]
$\mathrm{var}(S_T) = \entero{\cnueve} \cdot \decimalx{\cdoce} \cdot \left( \decimalx{\ctrece} - 1 \right)$\\[0.3em]
$\mathrm{var}(S_T) = \decimal{\ccatorce} \cdot \left( \decimalx{\cquince}  \right)$\\[0.3em]
$\mathrm{var}(S_T) = \decimal{\cdieciseis}$\\[0.3em]
$\mathrm{sd}(S_T)^2 = \mathrm{var}(S_T)$\\[0.3em]
$\mathrm{sd}(S_T) = \sqrt{\mathrm{var}(S_T)}$\\[0.3em]
$\mathrm{sd}(S_T) = \sqrt{\decimal{\cdieciseis}}$\\[0.3em]
$\mathrm{sd}(S_T) = \decimal{\cdiecisiete}$\\[0.3em]
\cajaverdeletra{Por lo tanto, la desviación estandard de la distribución es \decimal{\cdiecisiete}.}
\end{flushleft}

\subpregunta{Parte d)}
\noindent Para determinar el intervalo de confianza del 95\% utilizamos la fórmula:
\begin{flushleft}
\formula{\intervaloconf}\\[0.5em]

$IC = \mathbb{E}(S_T) \pm Z_{\alpha/2} \mathrm{sd}(S_T))$\\[0.3em]
$IC = \decimal{\cocho} \pm 1.96 \cdot \decimal{\cdiecisiete}$\\[0.3em]
$IC = \decimal{\cocho} \pm \decimal{\duno}$\\[0.3em]
$IC = \left[\decimal{\dtres},\decimal{\ddos}\right]$\\[0.3em]
\cajaverdeletra{Se puede decir con un 95\% de confianza que el precio de la acción estará entre \dineross{\dtres} y \dineross{\ddos}
como precio final a los {\Tcuatro} años}
\end{flushleft}

\pregunta{ Hull 14.20}
\justify
Suponga que \( x \) es el rendimiento al vencimiento (yield to maturity) con capitalización continua de un bono cupón cero que paga \$1 en el tiempo \( T \). Se asume que \( x \) sigue el siguiente proceso estocástico:
\begin{equation} 
    dx = a\cdot( x_0 - x^2) \, dt + s x \, dz 
\end{equation}

Dónde \( a \), \( x_0 \) y \( s \) son constantes positivas, y \( dz \) es un proceso de Wiener. ¿Cuál es el proceso seguido por el precio del bono?

\subpregunta{Respuesta}

\ El precio del bono cupón cero que paga 1 en \( T \) está dado por:
\[
B(t) = e^{-x(T - t)}
\]

\noindent
El Lema de Itô dice que, si una variable \( x \) sigue un proceso estocástico \( dx = \mu dt + \sigma dz \), y \( f(x,t) \) es una función suficientemente suave, entonces:\\
\formula{\ito}\\

Aplicamos el Lema de Itô al proceso \( B(t) \), que depende de \( x \). Como:
\[
dx = a(x_0 - x)\,dt + s x\,dz
\]

Calculamos derivadas:
\begin{align*}
    \frac{\partial B}{\partial t} &= x e^{-x(T - t)} = x B(t) \\
    \frac{\partial B}{\partial x} &= - (T - t) B(t) \\
    \frac{\partial^2 B}{\partial x^2} &= (T - t)^2 B(t)
\end{align*}

Aplicamos Itô:
\begin{align*}
    dB &= \left[ \frac{\partial B}{\partial t} + \frac{\partial B}{\partial x} \cdot a(x_0 - x) + \frac{1}{2} \cdot \frac{\partial^2 B}{\partial x^2} \cdot (s x)^2 \right] dt + \frac{\partial B}{\partial x} \cdot s x \, dz \\
    &= B(t) \left[ x - (T - t) a (x_0 - x) + \frac{1}{2} (T - t)^2 s^2 x^2 \right] dt - (T - t) s x B(t) \, dz
\end{align*}
\cajaverdeletra{
El precio del bono sigue el siguiente proceso estocástico:
\[
dB = B(t) \left[ x- (T - t) a (x_0 - x) + \frac{1}{2} (T - t)^2 s^2 x^2 \right] dt - (T - t) s x B(t) \, dz
\]
}
\pregunta{ Hull 15.26}

Demuestre que las fórmulas de Black-Scholes-Merton para opciones call y put satisfacen la paridad put-call.

\subpregunta{Respuesta}
\noindent La fórmula de paridad put-call establece:\\
\formula{\putcall}\\[0.5em]

\noindent Utilizamos las fórmulas de Black-Scholes-Merton para call y put:
\begin{flushleft}
    
    \formula{\callbsm}\\[0.5em]
    \formula{\putbsm}\\[0.5em]
    Reemplazamos la call:\\[0.5em]
    $ S_0 + p = K e^{-rT} +S_0 \cdot \mathcal{N}(d_1)- K \cdot e^{-r \cdot (T)} \cdot \mathcal{N}(d_2)$\\ [0.3em]
    $ S_0 + p =S_0 \cdot \mathcal{N}(d_1)+ K e^{-rT} - K \cdot e^{-r \cdot (T)} \cdot \mathcal{N}(d_2)$\\ [0.3em]
    $ S_0 + p =S_0 \cdot \mathcal{N}(d_1)+ K e^{-rT} \cdot \left(1-\mathcal{N}(d_2)\right) $\\ [0.3em]
    Sabemos que \formula{\enemenosuno}, por lo tanto, podemos reemplazar:
    $ S_0 + p =S_0 \cdot \textcolor{purple}{\mathcal{N}(d_1)}+ K e^{-rT} \cdot \textcolor{red}{\left(1-\mathcal{N}(d_2)\right)} $\\ [0.3em]
    $ S_0 + p =S_0 \cdot \textcolor{purple}{\left(1-\mathcal{N}(-d_1)\right)}+ K e^{-rT} \cdot \textcolor{red}{\left(\mathcal{N}(-d_2)\right)} $\\ [0.3em]
    $ S_0 + p =S_0-S_0 \cdot \mathcal{N}(-d_1)+ K e^{-rT} \cdot \mathcal{N}(-d_2) $\\ [0.3em]
    $ S_0 + p =S_0+ \textcolor{grene}{K e^{-rT} \cdot \mathcal{N}(-d_2) -S_0 \cdot \mathcal{N}(-d_1)}$\\ [0.3em]
    $ S_0 + p =S_0+ \textcolor{grene}{p}$\\ [0.3em]

\end{flushleft}


\cajaverdeletra{Se verifica que \( c + K e^{-rT} = p + S_0 \), por lo tanto, se cumple la paridad put-call para las fórmulas de Black-Scholes-Merton.}
\end{document}
