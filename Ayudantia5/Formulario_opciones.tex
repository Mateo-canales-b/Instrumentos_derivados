\documentclass[12pt]{article}
\usepackage[utf8]{inputenc}     % Codificación
\usepackage{amsmath}            % Matemáticas
\usepackage{amssymb}            % Símbolos matemáticos
\usepackage{graphicx}           % Imágenes
\usepackage{tikz}               % Gráficos vectoriales
\usetikzlibrary{arrows.meta, positioning, calc}
\usepackage{pgfplots}           % Gráficos matemáticos
\pgfplotsset{compat=1.18}
\usepackage{xcolor}             % Colores
\usepackage{geometry}           % Márgenes
\usepackage{enumitem}           % Listas personalizadas
\usepackage{fancyhdr}           % Encabezados/pies de página
\usepackage{float}              % Posicionamiento de figuras/tablas
\usepackage{comment}            % Comentarios múltiples
\usepackage{booktabs}           % Tablas profesionales
\usepackage{hyperref}           % Hipervínculos
\usepackage{ragged2e}           % Alineación de texto


\renewcommand{\figurename}{Gráfico}  

\geometry{letterpaper, margin=1in}
\fancypagestyle{firststyle}{
    \fancyhf{}
    \lhead{\includegraphics[height=5cm]{../imagenes/logo.png}}
    \renewcommand{\headrulewidth}{0pt}
    }
    \pagestyle{plain}
    
    \definecolor{rojoudp}{RGB}{210,35,42}
\newcommand{\subrayadoRojo}[1]{{\color{rojoudp}\underline{\textcolor{black}{#1}}}}
\renewcommand{\thesection}{Pregunta \arabic{section}}
\setcounter{section}{1}
\newcommand{\pregunta}[1]{%
  \section*{\subrayadoRojo{\thesection  #1}}%
  \stepcounter{section}%
}
\vspace{-3em}
\begin{document}
%\begin{figure}
%    \vspace{-5em}    
%    \flushright
%    \includegraphics[height=4cm]{../Ayudantia1/imagenes/logo.png}\\[-3em]
%\end{figure}
%\begin{center}
%    {\LARGE \textbf{Formulario opciones}}\\[0.5em]
%    Curso: Instrumentos Derivados\\
%    Profesor: Francisco Rantul\\
%    Ayudante: Mateo Canales\\
%\end{center}
%\vspace{1pt}
%{\color{rojoudp}\hrule height 2pt}
%\vspace{10pt}
\newcommand{\arbol}{$p = \frac{e^{r\cdot \Delta t}-d}{u-d}$}
\newcommand{\neutral}{$f = e^{-r \cdot \Delta t}\cdot (p \cdot f_u+(1-p) \cdot f_d)$}
\newcommand{\putcall}{$S_0+p = K \cdot e^{-r \cdot T}+c$}
\newcommand{\ceroud}{$S_0u|d=S_0*(1+(subida|bajada)) $}
\newcommand{\neutrali}{$S_0d\cdot \Delta-f_d=S_0u\cdot \Delta-f_u$}
\newcommand{\portafolio}{$S_0\cdot \Delta-f=(S_0u\cdot \Delta-f_u) \cdot e^{-rT}$}
\newcommand{\calcud}{$u|d=\frac{S_0u|d}{S_0}$}
\newcommand{\callbsm}{$c =  S_0 \cdot N(d_1)- K \cdot e^{-r \cdot (T)} \cdot N(d_2) $}
\newcommand{\putbsm}{$p = K \cdot e^{-r \cdot (T)} \cdot N(-d_2) - S_0 \cdot N(-d_1)$}
\newcommand{\Duno }{$  d_1 = \frac{\ln(S_0 / K) + \left( r + \frac{\sigma^2}{2} \right) \cdot T}{\sigma \cdot \sqrt{T}}$}
\newcommand{\Ddos}{$d_2 = d_1 - \sigma \cdot \sqrt{T}$}
\newcommand{\dividendo}{$S_1 = S_0 - e^{-r \cdot t} \cdot div $}
\newcommand{\vp}{$vp =e^{-r \cdot t} \cdot valor futuro $}
\newcommand{\browniano}{$\ln (S_T) \sim \mathcal{N}\left(\ln (S_0) + (\mu - \sigma^2/2)\cdot T,\ \sigma^2 \cdot T\right)$ }
\newcommand{\standarizar}{$Z = \frac{\ln K - \mathbb{E}[\ln S_T]}{\text{desv.\ estándar}}$}
\arbol\\ \vspace{2em}
\neutral\\ \vspace{2em}
\putcall\\ \vspace{2em}
\ceroud\\ \vspace{2em}
\neutrali\\ \vspace{2em}
\portafolio\\ \vspace{2em}
\calcud\\ \vspace{2em}
\callbsm\\ \vspace{2em}
\putbsm\\ \vspace{2em}
\Duno\\ \vspace{2em}
\Ddos\\ \vspace{2em}
\dividendo\\ \vspace{2em}
\vp\\ \vspace{2em}
\browniano\\ \vspace{2em}
\standarizar\\ \vspace{2em}
\end{document}