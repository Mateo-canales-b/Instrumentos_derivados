\documentclass[12pt]{article}
\usepackage[utf8]{inputenc}
\usepackage{amsmath}
\usepackage{graphicx}
\usepackage{xcolor}
\usepackage{geometry}
\usepackage{enumitem}
\usepackage{fancyhdr}
\usepackage{titling}
\geometry{letterpaper, margin=1in}
\fancypagestyle{firststyle}{
    \fancyhf{}
    \lhead{\includegraphics[height=5cm]{../imagenes/logo.png}}
    \renewcommand{\headrulewidth}{0pt}
}
\pagestyle{plain}

\definecolor{rojoudp}{RGB}{210,35,42}
\newcommand{\subrayadoRojo}[1]{{\color{rojoudp}\underline{\textcolor{black}{#1}}}}
\newcommand{\formulasection}[2]{%
    \vspace{1em}
    \noindent
    {\large\bfseries\color{blue}#1}\\[-4em]
    \begin{center}
        \large
      \[
      #2
      \]
    \end{center}
    \vspace{1em}
}
\setlength{\droptitle}{-3em}
\begin{document}
\begin{figure}
    \vspace{-5em}    
    \flushright
    \includegraphics[height=4cm]{../imagenes/logo.png}\\[-3em]
\end{figure}
\begin{center}
    {\LARGE \textbf{Ayudantía Nº2: Valoración de Futuros y Forwards}}\\[0.5em]
    Curso: Instrumentos Derivados\\
    Profesor: Francisco Rantul\\
    Ayudante: Mateo Canales\\
    \date{31/03/2025}
\end{center}
\vspace{1pt}
{\color{rojoudp}\hrule height 2pt}
\vspace{10pt}

\section*{\subrayadoRojo{Pregunta 1}}
Suponga que la acción A tiene un precio de \$28 y sus flujos de caja esperados en el próximo periodo 
es de \$35,448 en el escenario bueno y de \$24,511 en el escenario malo. La acción B tiene un valor de 
\$12,019 y posee flujos esperados de \$14,788 en el escenario bueno y flujos de \$10,949 en el escenario malo. 

\begin{enumerate}[label=\textbf{\alph*)}]
\item   Asumiendo que no existen oportunidades de arbitraje, calcule cual sería la tasa libre de riesgo.
\textbf{HINT}: Asuma que las probabilidades neutrales al riesgo son de $\pi=0,5$ en cada escenario. 

\item   Comente intuitivamente qué cambia respecto de lo utilizado en el punto a) cuando hay oportunidades de arbitraje. 

\end{enumerate}

\section*{\subrayadoRojo{Pregunta 2}}
Se espera que una acción pague dividendos equivalentes a \$1 por acción en 4 meses y en 10 meses.
El precio de la acción hoy es de \$28, y la tasa cero libre de riesgo es de 0,07 anual (compuesta continua).
Un inversionista ha tomado una posición corta en un contrato forward sobre la acción a 12 meses.

\begin{enumerate}[label=\textbf{\alph*)}]
\item   ¿Cuál es el precio del forward y el valor del contrato inicial?
\item   9 meses después, el precio de la acción es de \$30 y la tasa libre de riesgo sigue siendo la misma.
 ¿Cuál es el precio del forward y el valor del contrato?
\item   En pandemia las empresas decidieron distribuir un alto porcentaje de sus utilidades como dividendos 
debido a las pocas oportunidades de inversión en nuevos proyectos. ¿Como influyó este shock en los precios 
forward acciones? ¿en base a lo anterior, de qué forma usted anticiparía una recuperación de la economía?
\end{enumerate}

\section*{\subrayadoRojo{Pregunta 3}}
Una firma importadora el día 25 de Agosto 2022 necesitaba realizar una cobertura de tipo de cambio 
para un año, el tipo de cambio se encontraba en \$683,2. Asuma convención 30/360 y que la empresa debe comprar dólares.

\begin{enumerate}[label=\textbf{\alph*)}]
\item Determine el precio forward a 360 días utilizando la siguiente información de curvas cero cupón:
\begin{table}[h!]
    \centering
    \caption{Curvas cero cupón al 25-08-2022}
    \begin{tabular}{|c|c|c|c|c|c|c|c|}
    \hline
    \textbf{Curva} & \textbf{1 Día} & \textbf{30 Días} & \textbf{60 Días} & \textbf{90 Días} & \textbf{180 Días} & \textbf{1 Año} & \textbf{2 Años} \\
    \hline
    CLP & 2{,}81\% & 3{,}01\% & 3{,}11\% & 3{,}16\% & 3{,}25\% & 3{,}56\% & 4{,}18\% \\
    UF  & 2{,}93\% & 3{,}10\% & 1{,}99\% & 1{,}18\% & 0{,}42\% & 0{,}89\% & 1{,}26\% \\
    USD & 2{,}66\% & 2{,}75\% & 2{,}84\% & 2{,}95\% & 3{,}26\% & 3{,}96\% & 5{,}35\% \\
    \hline
    \end{tabular}
\end{table}
\item	Suponga que 10 meses después el tipo de cambio se encuentra en \$708 y la firma quiere ver
la posibilidad de vender el contrato, ¿cuál sería el precio justo de venta de dicho contrato?
\begin{table}[h!]
    \centering
    \caption{Curvas cero cupón al 25-08-2023}
    \begin{tabular}{|c|c|c|c|c|c|c|c|}
    \hline
    \textbf{Curva} & \textbf{1 Día} & \textbf{30 Días} & \textbf{60 Días} & \textbf{90 Días} & \textbf{180 Días} & \textbf{1 Año} & \textbf{2 Años} \\
    \hline
    CLP & 1{,}64\% & 2{,}09\% & 2{,}64\% & 2{,}90\% & 3{,}01\% & 3{,}15\% & 3{,}77\% \\
    UF  & 1{,}20\% & 1{,}71\% & 1{,}80\% & 1{,}95\% & 2{,}05\% & 1{,}98\% & 1{,}79\% \\
    USD & 0{,}50\% & 0{,}88\% & 1{,}00\% & 1{,}32\% & 1{,}50\% & 2{,}00\% & 3{,}55\% \\
    \hline
    \end{tabular}
\end{table}
\item Calcule las ganancias o pérdidas(contable) 10 meses despuésde firmado el contrato en el 
    punto a). \textbf{HINT}: use las tasasde la tabla b
\end{enumerate}

\section*{\subrayadoRojo{Pregunta 4}}
Suponga que el precio spot del commodity de plata es actualmente igual a \$18,8 dólares  
la onza. Los costos de almacenamiento son iguales a \$0,4 por año la onza, pagaderos por
 trimestres vencidos. La estructura de tasas de interés es plana con una tasa cero libre 
 de riesgo del 4\% anual compuesto continuo.
 \begin{enumerate}[label=\textbf{\alph*)}]
    \item   Se le pide calcular cual debería ser el precio de futuros de plata, con entrega a 9 meses plazo.
    \item 	Explique que ocurre si el precio de futuros de plata con entrega a 9 meses,
     efectivamente observado en el mercado es de \$19,8 dólares la onza.
    \item suma que existe una baja demanda por el commodity de plata, por lo tanto,
     las empresas están almacenando altos niveles de inventarios. 
     Explique qué ocurre con el rendimiento de conveniencia.

\end{enumerate}


\end{document}