\documentclass[12pt]{article}
\usepackage[utf8]{inputenc}
\usepackage{amsmath}
\usepackage{graphicx}
\usepackage{xcolor}
\usepackage{geometry}
\usepackage{enumitem}
\usepackage{fancyhdr}
\usepackage{titling}
\geometry{letterpaper, margin=1in}
\fancypagestyle{firststyle}{
    \fancyhf{}
    \lhead{\includegraphics[height=5cm]{../imagenes/logo.png}}
    \renewcommand{\headrulewidth}{0pt}
}
\pagestyle{plain}

\definecolor{rojoudp}{RGB}{210,35,42}
\newcommand{\subrayadoRojo}[1]{{\color{rojoudp}\underline{\textcolor{black}{#1}}}}
\newcommand{\formulasection}[2]{%
    \vspace{1em}
    \noindent
    {\large\bfseries\color{blue}#1}\\[-4em]
    \begin{center}
        \large
      \[
      #2
      \]
    \end{center}
    \vspace{1em}
}
\setlength{\droptitle}{-3em}
\begin{document}
\begin{figure}
    \vspace{-5em}    
    \flushright
    \includegraphics[height=4cm]{../imagenes/logo.png}\\[-3em]
\end{figure}
\begin{center}
    {\LARGE \textbf{Ayudantía Nº2: Valoración de Futuros y Forwards}}\\[0.5em]
    Curso: Instrumentos Derivados\\
    Profesor: Francisco Rantul\\
    Ayudante: Mateo Canales\\
    \date{31/03/2025}
\end{center}
\vspace{1pt}
{\color{rojoudp}\hrule height 2pt}
\vspace{10pt}

\section*{\subrayadoRojo{Pregunta 1}}
Suponga que la acción A tiene un precio de \$28 y sus flujos de caja esperados en el próximo periodo 
es de \$35,448 en el escenario bueno y de \$24,511 en el escenario malo. La acción B tiene un valor de 
\$12,019 y posee flujos esperados de \$14,788 en el escenario bueno y flujos de \$10,949 en el escenario malo. 

\begin{enumerate}[label=\textbf{\alph*)}]
\item   Asumiendo que no existen oportunidades de arbitraje, calcule cual sería la tasa libre de riesgo.
\textbf{HINT}: Asuma que las probabilidades neutrales al riesgo son de $\pi=0,5$ en cada escenario. 

\item   Comente intuitivamente qué cambia respecto de lo utilizado en el punto a) cuando hay oportunidades de arbitraje. 

\end{enumerate}

\section*{\subrayadoRojo{Pregunta 2}}
Se espera que una acción pague dividendos equivalentes a \$1 por acción en 4 meses y en 10 meses.
El precio de la acción hoy es de \$28, y la tasa cero libre de riesgo es de 0,07 anual (compuesta continua).
Un inversionista ha tomado una posición corta en un contrato forward sobre la acción a 12 meses.

\begin{enumerate}[label=\textbf{\alph*)}]
\item   ¿Cuál es el precio del forward y el valor del contrato inicial?
\item   9 meses después, el precio de la acción es de \$30 y la tasa libre de riesgo sigue siendo la misma.
 ¿Cuál es el precio del forward y el valor del contrato?
\item   En pandemia las empresas decidieron distribuir un alto porcentaje de sus utilidades como dividendos 
debido a las pocas oportunidades de inversión en nuevos proyectos. ¿Como influyó este shock en los precios 
forward acciones? ¿en base a lo anterior, de qué forma usted anticiparía una recuperación de la economía?
\end{enumerate}

\section*{\subrayadoRojo{Pregunta 3}}
Una firma importadora el día 25 de Agosto 2022 necesitaba realizar una cobertura de tipo de cambio 
para un año, el tipo de cambio se encontraba en \$683,2. Asuma convención 30/360 y que la empresa debe comprar dólares.

\begin{enumerate}[label=\textbf{\alph*)}]
\item Determine el precio forward a 360 días utilizando la siguiente información de curvas cero cupón:
\begin{table}[h!]
    \centering
    \caption{Curvas cero cupón al 25-08-2022}
    \begin{tabular}{|c|c|c|c|c|c|c|c|}
    \hline
    \textbf{Curva} & \textbf{1 Día} & \textbf{30 Días} & \textbf{60 Días} & \textbf{90 Días} & \textbf{180 Días} & \textbf{1 Año} & \textbf{2 Años} \\
    \hline
    CLP & 2{,}81\% & 3{,}01\% & 3{,}11\% & 3{,}16\% & 3{,}25\% & 3{,}56\% & 4{,}18\% \\
    UF  & 2{,}93\% & 3{,}10\% & 1{,}99\% & 1{,}18\% & 0{,}42\% & 0{,}89\% & 1{,}26\% \\
    USD & 2{,}66\% & 2{,}75\% & 2{,}84\% & 2{,}95\% & 3{,}26\% & 3{,}96\% & 5{,}35\% \\
    \hline
    \end{tabular}
\end{table}
\item	Suponga que 10 meses después el tipo de cambio se encuentra en \$708 y la firma quiere ver
la posibilidad de vender el contrato, ¿cuál sería el precio justo de venta de dicho contrato?.
\begin{table}[h!]
    \centering
    \caption{Curvas cero cupón al 25-08-2023}
    \begin{tabular}{|c|c|c|c|c|c|c|c|}
    \hline
    \textbf{Curva} & \textbf{1 Día} & \textbf{30 Días} & \textbf{60 Días} & \textbf{90 Días} & \textbf{180 Días} & \textbf{1 Año} & \textbf{2 Años} \\
    \hline
    CLP & 1{,}64\% & 2{,}09\% & 2{,}64\% & 2{,}90\% & 3{,}01\% & 3{,}15\% & 3{,}77\% \\
    UF  & 1{,}20\% & 1{,}71\% & 1{,}80\% & 1{,}95\% & 2{,}05\% & 1{,}98\% & 1{,}79\% \\
    USD & 0{,}50\% & 0{,}88\% & 1{,}00\% & 1{,}32\% & 1{,}50\% & 2{,}00\% & 3{,}55\% \\
    \hline
    \end{tabular}
\end{table}
\item Calcule las ganancias o pérdidas(contable) 10 meses despuésde firmado el contrato en el 
    punto a). \textbf{HINT}: use las tasas de la tabla b.
\end{enumerate}

\section*{\subrayadoRojo{Pregunta 4}}
Suponga que el precio spot del commodity de plata es actualmente igual a \$18,8 dólares  
la onza. Los costos de almacenamiento son iguales a \$0,4 por año la onza, pagaderos por
trimestres vencidos. La estructura de tasas de interés es plana con una tasa cero libre 
de riesgo del 4\% anual compuesto continuo.
\begin{enumerate}[label=\textbf{\alph*)}]
    \item   Se le pide calcular cual debería ser el precio de futuros de plata, con entrega a 9 meses plazo.
    \item 	Explique que ocurre si el precio de futuros de plata con entrega a 9 meses,
    efectivamente observado en el mercado es de \$19,8 dólares la onza.
    \item suma que existe una baja demanda por el commodity de plata, por lo tanto,
    las empresas están almacenando altos niveles de inventarios. 
    Explique qué ocurre con el rendimiento de conveniencia.
    
\end{enumerate}

\section*{\subrayadoRojo{Pregunta 5}}
    Suponga que LAN está preocupada por las variaciones del precio del petróleo durante los siguientes meses,
    específicamente los siguientes 6 meses, ya que el combustible que ocupa la flota de aviones ocupa un importante 
    ítem de costo en el presupuesto de la firma, LAN sabe que necesitara 1.060.000 litros de combustible en 6 meses
    más para sus operaciones nacionales, el actual precio de combustible para aviones está en USD 65 el barril (159 litros). 
    El gran problema que posee LAN es que el precio del combustible depende directamente del precio del Barril de Petróleo 
    en el mercado internacional el cual es actualmente de USD 48,6 y solo existen derivados sobre petróleo y no sobre combustible.

    \begin{enumerate}[label=\textbf{\alph*)}]
        \item Calcule el precio teórico de un futuro sobre petróleo a 6 meses si existe un costo trimestral de almacenaje y 
        envió de USD 0,5 y la tasa libre de riesgo relevante es de 3\% anual (compuesta continua).
        \item Suponga que la desviación estándar de los cambios semestrales en los precios del combustible es USD 1.231 y 
        que la desviación estándar de los cambios semestrales en el precio del futuro del petróleo es de USD 1.285, y el coeficiente 
        de correlación entre estos cambios es de 0.94. Calcule el ratio de cobertura (hedge ratio) óptimo para un contrato a 6 meses 
        y explique cómo LAN podría ocupar el número calculado.
        \item Calcule cuantos contratos de petróleo necesitara LAN para la cobertura de combustible si cada contrato es de 150 barriles.
        \item Cuál sería el resultado de la estrategia de LAN si en 6 meses más el precio del petróleo fuera de USD 57,5 el barril y el de combustible de USD 75,5.
        \item Cuál sería el resultado de la estrategia de LAN si en 6 meses más el precio del petróleo fuera de USD 41,5 el barril y el de combustible de USD 58,1.
\end{enumerate}

\section*{\subrayadoRojo{Pregunta 6}}
El gerente de un fondo de inversión en acciones nacionales mantiene una cartera valorada en 18 millones de US\$ con un Beta de 
1,25 en septiembre del 2022; el gerente está preocupado sobre el comportamiento del mercado durante los próximos once meses y 
piensa en utilizar contratos de futuro a doce meses sobre el índice IPSA para cubrir el riesgo. El nivel del índice el 01/09/2022 
es de 4.055 y un contrato de futuros es sobre 250 veces el índice. La tasa de interés libre de riesgo es de 3\% anual, la tasa de 
rendimiento por dividendos sobre las acciones del índice es de 0.8\% anual y el tipo de cambio de dicho día fue de \$800  Se pide:
\begin{enumerate}[label=\textbf{\alph*)}]
    \item Cuál es el precio teórico del futuro para el contrato de futuros de 12 meses.
    \item Qué posición en contratos de futuros debe tomar el gerente del Fondo, para realizar una cobertura óptima a lo largo de los próximos 12 meses.
    \item Calcule el efecto de su estrategia sobre el rendimiento del Fondo que administra el gerente, si el nivel del mercado en septiembre del 2023 del índice es 4.144.
    \item Calcule el efecto de su estrategia sobre el rendimiento del Fondo que administra el gerente, si el nivel del mercado en septiembre del 2023 del índice es 3.615.

\end{enumerate}
    
\end{document}